\documentclass{article}

%Configuraciones de idioma del documento
\usepackage[spanish]{babel}
%Configurar entrada {usar caracteres desde el teclado como á}
\usepackage[utf8]{inputenc}
%Fuente que pueda renderizar caracteres en nuestro idioma
\usepackage[T1]{fontenc}
%Fuente compatible
\usepackage{lmodern}

\begin{document}

\title{Guía de Redes de Computadoras Unidad 1}
\author{Antonio Emiko Ochoa Adame}
\maketitle

\section{Definiciones}

Las redes se basan en protocolos.

Sin uno o más de los 5 elementos de la comunicación, esta no puede existir.

\textbf{Modulación de banda base}.- Pasar de señal digital a señal analógica.

\textbf{Medio guiado}.- La información va de origen a destino sin pasar los
límites del medio.

\section{Factores que afectan la comunicación}

\subsection{Factores externos}

\begin{itemize}
	\item
	\textbf{Diafonía}.- Empalme de señales.
	\item
	Calidad de la ruta
	\item
	Calidad de veces que el mensaje cambia de forma.
\end{itemize}

\subsection{Factores internos}

\begin{itemize}
	\item
	Tamaño del mensaje.
	\item
	Importancia del mensaje.
\end{itemize}

\section{Símbolos/dispositivos de las redes de datos}

\textbf{Switch}.- Importante en interconexión. Se usa para hacer una red. Conecta
más de dos dispositivos entre sí.

\textbf{HUB}.- Es concentrado de señales y las transmite (ya no se utiliza tanto).

El switch hace lo mismo que el HUB, pero administra a quién le manda el mensaje;
esto lo hace a través del direccionamiento.

Una conexión inalámbrica siempre proviene de una conexión alámrica.

\section{Protocolos}

\textbf{Protocolo}.- Reglas de transmisión de un mensaje.

\textbf{Protocolos}.- Son reglas que rigen la comunicación de los datos.

Se utilizan a través de estándares.

\section{Modelo OSI}

\begin{itemize}
	\item
	(7)Aplicación
	\item
	(6)Presentación
	\item
	(5)Sesión
	\item
	(4)Transporte
	\item
	(3)Red
	\item
	(2)Enlace
	\item
	(5)Física
\end{itemize}

\textbf{Aplicación}
\vspace{1em}

Se genera el mensaje.

\textbf{Presentación}
\vspace{1em}

Determina el formato de los datos.

\textbf{Sesión}
\vspace{1em}

Abre sesión entre dos dispositivos.

\textbf{Transporte}
\vspace{1em}

Control de flujo y segmentación del mensaje.

\textbf{Red}
\vspace{1em}

Verifica direccionamiento IP y determina la ruta.

\textbf{Enlace}
\vspace{1em}

Determina con qué tecnología se enviará un mensaje (UTP, coaxial, Wi-Fi).
\vspace{1em}

\textbf{Física}
\vspace{1em}

Intraestructua; cableado, bits, modulación.
\vspace{1em}

\section{Otros}

\textbf{MAC}
\vspace{1em}

(Media Access Control). Se compone de 48 bits y su valor está representado en
hexadecimal.

Es un ID único asignado a un NIC (Controlador de Interfaz de Red).

Los primeros 24 bits es de OUI y los últimos 24 son del número de serie del
fabricante.

El proceso del modelo OSI se llama encapsulamiento y desencapsulamiento.

\section{Clasificación de Redes}

\begin{itemize}
	\item
		Por alcance.

		\begin{itemize}
			\item
			PAN
			\item
			LAN
			\item
			WAN
			\item
			CAN
			\item
			MAN
		\end{itemize}
	\item
		Por relación funcional.

		\begin{itemize}
			\item
			Cliente-servidor
			\item
			peer-to-per
		\end{itemize}
	\item
		Por topología.

		\begin{itemize}
			\item
			Red en bus
			\item
			Red de anillo
			\item
			Red de estrella
			\item
			Red en malla (todos conectados con todos)
			\item
			Red en árbol
		\end{itemize}
	\item
		Por modo de direccionamiento/modo de transmisión de datos.

		\begin{itemize}
			\item
			Simplex
			\item
			Half-duplex o semi-duplex (Ambos sentidos pero
			no al mismo tiempo)
			\item
			Full-duplex (Ambos sentidos pero y al mismo tiempo)
		\end{itemize}
	\item
		Por grado de autenticación.

		\begin{itemize}
			\item
			Red privada
			\item
			Red de acceso público
		\end{itemize}
	\item
		Por grado de difusión.

		\begin{itemize}
			\item
			Intranet
			\item
			Internet
		\end{itemize}
	\item
		Por servicio o función

		\begin{itemize}
			\item
			Red comercial
			\item
			Red educativa
			\item
			Red para procesado de datos
		\end{itemize}
\end{itemize}

\section{Cliente-servidor}

\textbf{Listen}.- Primitiva de servicio.
\textbf{Request}.- Primitiva de servicio.

EL servidor está a la escucha mientras el cliente hace solicitudes.

Ejemplos de servidores:

\begin{itemize}
	\item
	Servidores de correo
	\item
	Servidores web
	\item
	Servidores de ftp
\end{itemize}

\textbf{Disclaimer}: La finalidad de este documento es servir como apoyo de estudio.
El autor de la versión original de este documento no se hace responsable del
uso indebido del mismo.

\end{document}
