\documentclass{article}

%Configuraciones de idioma del documento
\usepackage[spanish]{babel}
%Configurar entrada {usar caracteres desde el teclado como á}
\usepackage[utf8]{inputenc}
%Fuente que pueda renderizar caracteres en nuestro idioma
\usepackage[T1]{fontenc}
%Fuente compatible
\usepackage{lmodern}

\usepackage{spverbatim}

\begin{document}

\title{Tarea 6}
\author{Antonio Emiko Ochoa Adame}
\maketitle

1.- Escriba una expresión regular para identificar palabras entre 4 y 15
caracteres, formadas por letras minúsculas exclusivamente.

\begin{spverbatim}
R = /\b[a-z]{4,15}\b/g
\end{spverbatim}
\vspace{1em}

2.- Escriba una expresión regular para identificar las direcciones de correo,
con las siguientes características:

\begin{itemize}
	\item
	Antes de la arroba puede haber tres o más caracteres alfanuméricos,
	incluyendo el punto.
	\item
	La arroba es obligatoria.
	\item
	Después de la arroba puede haber varios (al menos uno) grupos de dos o
	más caracteres alfanuméricos, separados por un punto.
	\item
	Para terminar debe haber una extensión de dos a cuatro caracteres.
\end{itemize}

\begin{spverbatim}
R = /[\w.]{3,}@(\w{2,}\.)+\w{2,4}/g
\end{spverbatim}
\vspace{1em}

3.- Escriba una expresión regular para reconocer fechas en el formato
\textbf{dd/mm/aaaa}, de acuerdo con los siguientes criterios:

\begin{itemize}
	\item
	\textbf{dd} entre 01 y 30
	\item
	\textbf{mm} entre 01 y 12
	\item
	\textbf{aaaa} que contenga cuatro dígitos
\end{itemize}

\begin{spverbatim}
R = /\b(0[1-9]|[12]\d|30)\/(0[1-9]|1[012])\/\d{4}\b/g
\end{spverbatim}
\vspace{1em}

4.- Escriba una expresión regular para reconocer una hora en el formato
\textbf{HH:mm:ss}, en la que los valores de \textbf{HH} estén entre 00 y 24 y
los valores de \textbf{mm} y \textbf{ss} entre 00 y 59.

\begin{spverbatim}
R = /\b([01]\d|2[0-4])(:(0\d|[1-5]\d)){2}\b/g
\end{spverbatim}
\vspace{1em}

5.- Escriba una expresión regular que sirva para reconocer una contraseña que
esté formada por 8 caracteres o más y que contenga uno o más dígitos, una o
más letras minúsculas y una o más mayúsculas.

\begin{spverbatim}
R = /(?=\w{8,})(?=.*\d+.*)(?=.*[a-z]+.*)(?=.*[A-Z]+.*).*/g
\end{spverbatim}
\vspace{1em}

Sin caracteres especiales:

\begin{spverbatim}
R = /(?=\w{8,})(?=\w*\d+\w*)(?=\w*[a-zA-Z]+\w*)(?=\w*[A-Z]+\w*)\w*/g
\end{spverbatim}

\end{document}
