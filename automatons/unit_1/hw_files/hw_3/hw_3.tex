\documentclass{article}

%Configuraciones de idioma del documento
\usepackage[spanish]{babel}
%Configurar entrada {usar caracteres desde el teclado como á}
\usepackage[utf8]{inputenc}
%Fuente que pueda renderizar caracteres en nuestro idioma
\usepackage[T1]{fontenc}
%Fuente compatible
\usepackage{lmodern}

\begin{document}

\title{Tarea 3}
\author{Antonio Emiko Ochoa Adame}
\maketitle

\begin{enumerate}
	\item Dado el lenguaje $L = \{$sa, ro$\}$, obtenga $L^3$.

		Respuesta: $L^3 = \{$sa, ro, sasa, saro, rosa, roro, sasasa,
		sasaro, sarosa, saroro, rosasa, rosaro, rorosa, rororo$\}$.
	\item Dado el lenguaje $L = \{\varepsilon, $ ab$\}$, obtenga $L^0$,
		$L^1$, $L^2$, $L^3$, $L^4$.

		Respuesta:$L^0 = \{\varepsilon\}$, $L^1 = \{\varepsilon, $ ab$\}$,
		$L^2 = \{\varepsilon, $ ab$, $ abab$\}$,
		$L^3 = \{\varepsilon, $ ab$, $ abab$, $ ababab$\}$,
		$L^4 = \{\varepsilon, $ ab$, $ abab$, $ ababab$, $ abababab$\}$.
	\item Sean A $= \{$a$\}$ y B $= \{$b$\}$, indique cuáles son las cadenas
		que forman los siguentes lenguajes: A$^*$B, AB$^*$ y $($AB$)^*$.

		Respuesta: A$^*$B$ = \{$b$, $ ab$, $ aab$, $ aaab$, ...\}$,
		AB$^* = \{$a$, $ ab$, $ abb$, $ abbb$, ...\}$.
		$($AB$)^* = \{\varepsilon, $ ab$, $ abab$, $ ababab$, ...\}$.
	\item Dado los lenguajes: A $= \{011, 001, 11\}$ y B $= \{11, 110\}$
		sobre el alfabeto $\Sigma = \{0, 1\}$, obtenga los lenguajes
		que resultan de las operaciones siguientes: $($A$\cap$B$)^*$,
		$($A$\oplus$B$)^R$, $($B$-$A$)^+$, BA.

		Respuesta:
		$($A$\cap$B$)^* = \{\varepsilon,$ 11$,$ 1111$,$ 11111111$,...\}$,
		$($A$\oplus$B$)^R = \{$ 110$,$ 100$,$ 001$\}$,
		$($A$-$B$)^+ = \{$ 110$,$ 110110$,$ 110110110$...\}$.
		AB $= \{$ 11011$,$ 11001$,$ 1111$, $ 110011$, $110001$\}$.
	\item Dado los lenguajes: A $= \{101, 01, 010\}$ y B $= \{10, 010\}$
		sobre el alfabeto $\Sigma = \{0, 1\}$, obtenga los lenguajes
		que resultan de las operaciones siguientes: $($A$\cap$B$)^3$,
		$($A$^R\oplus$B$^R)$, $($B$-$A$)^*$, BA.

		Respuesta: $($A$\cap$B$)^3 = \{$010010010$\}$,
		$($A$^R\oplus$B$^R) = \{$01$, $10$, $010$\}$,
		$($B$-$A$)^* = \{$10$, $1010$, $101010$...\}$,
		BA$)^* = \{$10101$,$ 1001$,$ 10010$,$ 010101$,$ 01001$,$ 010010$\}$.
	\item Dado los lenguajes: A $= \{\varepsilon, 0, 10, 11\}$ y B $=
		\{\varepsilon, 1, 01, 11\}$
		sobre el alfabeto $\Sigma = \{0, 1\}$, obtenga los lenguajes
		: AB, BA, A$\cup$B,
		A$\cap$B, A$-$B, B$-$A, A$^*$, B$^2$ y A$\oplus$B.

		Respuesta: AB $= \{\varepsilon,$ 1$,$ 01$,$ 11$,$ 0$,$ 001$,
		$ 011$,$ 10$,$ 101$,$ 1001$,$ 1011$,$ 111$,$ 1101$,$ 1111$,\}$,
		BA $= \{\varepsilon,$ 0$,$ 10$,$ 11$,$ 1$,$ 110$,
		$ 111$,$ 01$,$ 010$,$ 0110$,$ 0111$,$ 1110$,$ 1111$\}$,
		A$\cup$B $= \{\varepsilon,$ 0$,$ 10$,$ 11$,$ 1$,$ 01$\}$,
		A$\cap$B $= \{\varepsilon,$ 11$\}$,
		A$-$B $= \{$0$,$ 10$\}$,
		B$-$A $= \{$1$,$ 01$\}$,
		A$^* = \{\varepsilon,$ 0$,$ 10$,$ 11$,$ 00$,$ 010$,$ 011$,
		$ 100$,$ 1010$,$ 1011$,$ 110$,$ 1110$,$ 1111$, ...\}$,
		B$^2 = \{\varepsilon,$ 1$,$ 01$,$ 11$,$ 011$,$ 0101$,$ 0111$,
		$ 1101$,$ 1111$\}$,
		A$\oplus$B $= \{$0$,$ 10$,$ 1$,$ 01$\}$.
	\item Dado los lenguajes: A $= \{\varepsilon\}$, B $=
		\{$aa$,$ ab$,$ bb$\}$, C $= \{\varepsilon,$ aa$,$ ab$\}$ y
		D $= \o$ obtener los lenguajes:
		A$\cup$B, A$\cup$C, A$\cup$D, A$\cap$B, A$\cap$D, B$\cap$C,
		B$\cup$D y C$\cap$A.

		Respuesta: A$\cup$B $= \{\varepsilon, $ aa$,$ ab$,$ bb$\}$,
		A$\cup$C $= \{\varepsilon, $ aa$,$ ab$\}$,
		A$\cup$D $= \{\varepsilon\}$,
		A$\cap$B $= \o$,
		A$\cap$D $= \o$,
		B$\cap$C $= \{$aa$,$ ab$\}$,
		B$\cup$D $= \o$,
		C$\cap$A $= \varepsilon$.
	\item Dado los lenguajes: A $= \{$ab$,$ b$, $ cb$\}$ y B $=
		\{$a$,$ ba$\}$ obtener los
		lenguajes que resultan de las operaciones de lenguajes:
		$($A$\cup$B$^2)$, $($B$\cup$A$)^R$, $($AB$)$, $($A$^2\cap$BA$)$,
		$($A$\oplus$B$^R)$ y $($A$^R-$B$)^2$.

		Respuesta: $($A$\cup$B$^2) = \{$ab$,$ b$,$ cb$,$ aa$,$ aba$,
		$ baa$,$ baba$\}$,
		$($B$\cup$A$)^R = \{$ba$,$ b$,$ bc$,$ a$,$ ab$\}$,
		$($B$\cup$A$)^R = \{$ba$,$ b$,$ bc$,$ a$,$ ab$\}$,
		$($BA$) = \{$aba$,$ abba$,$ ba$,$ bba$,$ cba$,$ cbba$\}$,
		$($A$^2\cap$BA$) = \{$bb$,$ bab$,$ bcb$\}$,
		$($A$\oplus$B$^R) = \{$b$,$ cb$,$ a$\}$,
		$($A$^R-$B$)^2 = \{$bb$,$ bbc$,$ bcb$,$ bcbc$\}$.
	\item Dado los lenguajes A $= \{$01$,$ 11$\}$ y B $= \{$011$,$ 101$,$ 11$\}$
		obtener los lenguajes que resultan de las operaciones:
		$($A$\cup$B$)^R$, $($B$-$A$)^2$, $($A$-$B$)^+$, $($A$\cap$B$)^*$,
		A$^R$B.

		Respuesta: $($A$\cup$B$)^R = \{$10$,$ 11$,$ 110$,$ 101$\}$,
		$($B$-$A$)^2 = \{$011011$,$ 011101$,$ 101011$,$ 101101$\}$,
		$($A$-$B$)^+ = \{$01$,$ 0101$,$ 010101$, ...\}$,
		$($A$\cap$B$)^* = \{$01$,$ 0101$,$ 010101$, ...\}$,
		$($A$^R$B$) = \{$10011$,$ 10101$,$ 1011$,$ 11011$,$11101$,$ 1111$\}$.
	\item Responda \textbf{V}erdadero o \textbf{F}also según corresponda:
		\begin{enumerate}
			\item Para todo lenguaje $L$ se cumple que: $\o \cdot
				L = L$.

			Respuesta: \textbf{Falso}
			\item Para todo lenguaje $L$ infinito, se cumple que
				$L^c$ es finito.

			Respuesta: \textbf{Falso}
			\item Para todo lenguaje L regular, entonces
				$\varepsilon \notin L^+$.

			Respuesta: \textbf{Verdadero}
			\item La cerradura de Kleene del lenguaje vacío $\o$
				es igual a $\varepsilon$.

			Respuesta: \textbf{Verdadero}
			\item La cerradura de Kleene cualquier lenguaje $L$
				es infinita.

			Respuesta: \textbf{Verdadero}
			\item El lenguaje universal de cualquier alfabeto
				$\Sigma$ siempre es infinito.

			Respuesta: \textbf{Verdadero}
		\end{enumerate}
\end{enumerate}

\end{document}
