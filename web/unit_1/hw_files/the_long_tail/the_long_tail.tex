\documentclass{article}

%Configuraciones de idioma del documento
\usepackage[spanish]{babel}
%Configurar entrada {usar caracteres desde el teclado como á}
\usepackage[utf8]{inputenc}
%Fuente que pueda renderizar caracteres en nuestro idioma
\usepackage[T1]{fontenc}
%Fuente compatible
\usepackage{lmodern}

\usepackage[top=2cm, left=2cm, right=2cm, bottom=2cm]{geometry}

\begin{document}

\title{The Long Tail}
\author{Antonio Emiko Ochoa Adame}
\maketitle

\section{Conclusiones}

\paragraph{}
Como aporte principal se habla sobre que, anteriormente, las industrias de
entretenimiento en general siempre trataron los hits o los ``best sellers''
como la fuente de ingreso más importante y por la que había que velar siempre.
Entonces para que la gente pudiera generar sus propios gusto en música, por
ejemplo, tenían que empezar escuchando los grandes éxitos de la música mainstream
para después empezar a encontrar sus propios gustos.

\paragraph{}
Una razón por la cual se daba la situación anterior es que no había mucho espacio
físico para poner cada uno de los contenidos que iban apareciendo en la industria;
sería físicamente imposible a añadir a cada nuevo artista o poteciales
éxitos, así que solo aquellos que eran muy populares eran los que predominaban
en la industria y es lo que se consumía.

\paragraph{}
Con la llegada de internet todo cambió; desde la forma en la que es consumido
y comprado el contenido de entretenimiento hasta la forma de hacerle llegar el
contenido a los clientes. Esto hizo posible que una gran variedad contenido
saliera a la luz y que se puediesen aceptar por todas las personas con
diferentes gustos.

\paragraph{}
La mayoría de las personas tienen un mal concepto sobre como funciona la industria
del entretenimiento, o al menos un concepto que cambió con el paso del tiempo;
se piensa que en la industria de la música se puede aplicar el principio de Pareto,
esto querría decir que, en palabras sencillas, el 80 porciento del total de las
ventas se debe solamente al 20 porciento del total de los clientes.
Se puede decir que esto no aplica, al menos en esta industria, debido a que el
contenido que está siendo consumido es, de hecho, el 99 porciento del total como
se menciona en \cite{tlt}. Entonces hay personas que sí consumen contenido
considerado como no ``tradicional''; el porcentaje que lo consume no es muy alto,
sin embargo, lo que importa es que está siendo consumido.
Entonces, debido a eso, las gente normalmente piensa que ``si no es un hit,
no vale la pena que exista''.

\paragraph{}
Los ejecutivos de las grandes compañías descubrieron que los ``misses'', es
decir, aquellas piezas de contenido que no han sido puestas a la venta porque
no son tan populares como para pasar el filtro, también generan un
ingreso importante.

\paragraph{}
Tanto los hits como misses de hecho generan dinero, por lo tanto vale la pena
tenerlos en cuenta. ¿Entonces porqué es importante si no generan una gran cantidad
de ingresos?, bueno, la realidad es que esto produce que no exista más un
monopolio que esté controlado por el contenido más popular.  Por ejemplo,
incluso si no mucha gente escucha ciertos tracks en Rhapsody, sí hay gente
que los escucha y eso es suficiente.

\paragraph{}
A todo esto viene el término conocido como el \textbf{Long Tail} que, con mis propias
palabras, lo definiría como todos los productos que sí se venden, pero que
generan una pequeña cantidad de ingreso a comparación del contenido de
entetenimiento más popular.
Como se menciona en \cite{tlt}, Kevin Laws dice: ``La mayor cantidad de dinero proviene de las
ventas más pequeñas'' y esto aplica para cualquier tipo de contenido.
De hecho, el éxito de los grandes negocios en internet se basan en agregar el
\textbf{Long Tail} de una forma u otra. Por ejemplo, Google genera más dinero
de publicidad de los pequeños negocios.

\paragraph{}
Se dice que hay tres reglas para la nueva economía del entretenimiento:
\begin{itemize}
	\item
	Regla 1: Hacer todo disponible
	\item
	Regla 2: Recortar el precio a la mitad, no bajarlo
	\item
	Regla 3: Ayudar a encontrar contenido
\end{itemize}

\paragraph{}
La primera regla es, básicamente, añadir el \textbf{Long Tail} al mercado.

\paragraph{}
La segunda regla habla sobre distribuir el precio de un contenido de tal forma
que se genere un buen ingreso bajando el precio pero vendiendo mucha cantidad
de dicho contenido.

\paragraph{}
Y la regla número tres nos dice que al ayudar a los usuarios a encontrar contenido
asegura ventas, ya que a través de las recomendaciones, por ejemplo como lo hace
Amazon, se le pone ``a la mano'' contenido similar al que ha buscado y muy
probablemente esté interesado en consumirlo.

\paragraph{}
Y por último, mencionar que todo esto tiene una gran ventaja, ya que a diferencia
de la televisión, la radio o cualquier medio similar, a través de servicios por
internet puedes tener un servicio personalizado de ofertas y para cualquier tipo
de gusto.

\begin{thebibliography}{}
    \bibitem{tlt}
	C. Anderson. ``The Long Tail''.
	[online] Available at: https://www.wired.com/2004/10/tail/
	[Accessed 4 Feb. 2019].
\end{thebibliography}

\end{document}
