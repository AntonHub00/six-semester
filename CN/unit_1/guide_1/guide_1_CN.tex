\documentclass{article}

%Configuraciones de idioma del documento
\usepackage[spanish]{babel}
%Configurar entrada {usar caracteres desde el teclado como á}
\usepackage[utf8]{inputenc}
%Fuente que pueda renderizar caracteres en nuestro idioma
\usepackage[T1]{fontenc}
%Fuente compatible
\usepackage{lmodern}

\begin{document}

\title{Guía de Redes de Computadoras Unidad 1}
\author{Antonio Emiko Ochoa Adame}
\maketitle

\section{Definiciones}

Las redes se basan en protocolos.
\vspace{1em}

Sin uno o más de los 5 elementos de la comunicación, esta no puede existir.
\vspace{1em}

\textbf{Modulación de banda base}.- Pasar de señal digital a señal analógica.
\vspace{1em}

\textbf{Medio guiado}.- La información va de origen a destino sin pasar los
límites del medio.

\section{Factores que afectan la comunicación}

\subsection{Factores externos}

\begin{itemize}
	\item
	\textbf{Diafonía}.- Empalme de señales.
	\item
	Calidad de la ruta
	\item
	Calidad de veces que el mensaje cambia de forma.
\end{itemize}

\subsection{Factores internos}

\begin{itemize}
	\item
	Tamaño del mensaje.
	\item
	Importancia del mensaje.
\end{itemize}

\section{Símbolos/dispositivos de las redes de datos}

\textbf{Switch}.- Importante en interconexión. Se usa para hacer una red. Conecta
más de dos dispositivos entre sí.
\vspace{1em}

\textbf{HUB}.- Es concentrado de señales y las transmite (ya no se utiliza tanto).
\vspace{1em}

El switch hace lo mismo que el HUB, pero administra a quién le manda el mensaje;
esto lo hace a través del direccionamiento.
\vspace{1em}

Una conexión inalámbrica siempre proviene de una conexión alámrica.

\section{Protocolos}

\textbf{Protocolo}.- Reglas de transmisión de un mensaje.
\vspace{1em}

\textbf{Protocolos}.- Son reglas que rigen la comunicación de los datos.
\vspace{1em}

Se utilizan a través de estándares.

\section{Modelo OSI}

\begin{itemize}
	\item
	(7)Aplicación
	\item
	(6)Presentación
	\item
	(5)Sesión
	\item
	(4)Transporte
	\item
	(3)Red
	\item
	(2)Enlace
	\item
	(5)Física
\end{itemize}
\vspace{1em}

\textbf{Aplicación}

Se genera el mensaje.
\vspace{1em}

\textbf{Presentación}

Determina el formato de los datos.
\vspace{1em}

\textbf{Sesión}

Abre sesión entre dos dispositivos.
\vspace{1em}

\textbf{Transporte}

Control de flujo y segmentación del mensaje.
\vspace{1em}

\textbf{Red}

Verifica direccionamiento IP y determina la ruta.
\vspace{1em}

\textbf{Enlace}

Determina con qué tecnología se enviará un mensaje (UTP, coaxial, Wi-Fi).
\vspace{1em}

\textbf{Física}

Intraestructua; cableado, bits, modulación.
\vspace{1em}

\section{Otros}

\textbf{MAC}
\vspace{1em}

(Media Access Control). Se compone de 48 bits y su valor está representado en
hexadecimal.
\vspace{1em}

Es un ID único asignado a un NIC (Controlador de Interfaz de Red).
\vspace{1em}

Los primeros 24 bits es de OUI y los últimos 24 son del número de serie del
fabricante.
\vspace{1em}

El proceso del modelo OSI se llama encapsulamiento y desencapsulamiento.

\section{Clasificación de Redes}

\begin{itemize}
	\item
		Por alcance.

		\begin{itemize}
			\item
			PAN
			\item
			LAN
			\item
			WAN
			\item
			CAN
			\item
			MAN
		\end{itemize}
	\item
		Por relación funcional.

		\begin{itemize}
			\item
			Cliente-servidor
			\item
			peer-to-per
		\end{itemize}
	\item
		Por topología.

		\begin{itemize}
			\item
			Red en bus
			\item
			Red de anillo
			\item
			Red de estrella
			\item
			Red en malla (todos conectados con todos)
			\item
			Red en árbol
		\end{itemize}
	\item
		Por modo de direccionamiento/modo de transmisión de datos.

		\begin{itemize}
			\item
			Simplex
			\item
			Half-duplex o semi-duplex (Ambos sentidos pero
			no al mismo tiempo)
			\item
			Full-duplex (Ambos sentidos pero y al mismo tiempo)
		\end{itemize}
	\item
		Por grado de autenticación.

		\begin{itemize}
			\item
			Red privada
			\item
			Red de acceso público
		\end{itemize}
	\item
		Por grado de difusión.

		\begin{itemize}
			\item
			Intranet
			\item
			Internet
		\end{itemize}
	\item
		Por servicio o función

		\begin{itemize}
			\item
			Red comercial
			\item
			Red educativa
			\item
			Red para procesado de datos
		\end{itemize}
\end{itemize}

\section{Cliente-servidor}

\textbf{Listen}.- Primitiva de servicio.
\vspace{1em}

\textbf{Request}.- Primitiva de servicio.
\vspace{1em}

EL servidor está a la escucha mientras el cliente hace solicitudes.
\vspace{1em}

Ejemplos de servidores:

\begin{itemize}
	\item
	Servidores de correo
	\item
	Servidores web
	\item
	Servidores de ftp
\end{itemize}
\vspace{1em}

\section{Otros}

Las redes privadas tienen un segmeneto de direccionamiento privado.

\textbf{Routear un mensaje}.- A qué segmento de red estoy conectado.

\textbf{Direccionamiento}
\vspace{1em}

A.- 0-127

B.- 127-191

C.- 191-223

\section{Redes convergentes}

Son múltiples servicios sobre una misma red.
\vspace{1em}

Utlizan una misma infraestructura de red para utilizar todos los servicios.
\vspace{1em}

\textbf{Redes de información inteligentes}.- Basadas en dispositivos inteligentes.

\section{Arquitectura de una red}

Son los servicios y medios que conforman una red.
\vspace{1em}

Es lo que ``tiene que tener una red'':
\begin{itemize}
	\item
	Tolerancia a fallos
	\item
	Escalabilidad
	\item
	Calidad en el sevicio
	\item
	Seguridad
\end{itemize}
\vspace{1em}

\textbf{Tolerancia a fallos}
\vspace{1em}

Ejemplo: Si un router falla, buscar una ruta redundante/alterna
para evitar la pérdida de servicio.
\vspace{1em}

Esto ofrece una experiencia positiva al usuario; que nunca deje de tener
acceso a internet, por ejemplo.
\vspace{1em}

\textbf{Escalabilidad}

Que pueda crecer / tener conexiones adicionales siempre y cuando no disminuya
el rendimiento actual de la red.
\vspace{1em}

\textbf{Calidad del servicio}

Se admnistra dentro del router.
\vspace{1em}

Los servicios se pueden clasificar por grado de importancia.
\vspace{1em}

Ejemplo: Los streamings tienen prioridad alta o mayor sobre que un correo
electrónico, por ejemplo.
\vspace{1em}

\textbf{Seguridad}

Medidas de seguridad protegen la red de accesos no autorizados.

Ejemplos: contraseñas, inicios de sesión, etc.

Los admin. protegen las redes con hardware y software evitando así el acceso
físico a la red.
\vspace{1em}


\section{Modelos de protocolo y referencia}

\textbf{Modelo de referencia}.- Modelo OSI.
\vspace{1em}

\textbf{Modelo de protocolo}.- TCP/IP.
\vspace{1em}

El modelo de referencia (OSI) dice cómo debe ser la transmisión de un mensaje;
dan una idea de cómo debe fluir la información en la red sin involucrar procesos.
\vspace{1em}

El modelo de protocolo tiene 4 capas y se basa en protocolos; indica las reglas
exactas. Es la forma ``real'' de cómo se aplica la comunicación.
\vspace{1em}

Capas de TCP/IP:

\begin{itemize}
	\item
	Aplicación
	\item
	Transporte
	\item
	Internet
	\item
	Acceso a la red (genera la ruta del origen al destino)
\end{itemize}
\vspace{1em}

\textbf{Direccionamiento lógico}.- Direcciones IP.
\vspace{1em}

\textbf{Encapsulamiento}.- Proceso en el que la infoque pasa por la red en cada
etapa se le añade info que necesita.
\vspace{1em}

\section{PDU}

\textbf{PDU}.- Unidad de Datos de Protocolo (Protocol Data Unit).
\vspace{1em}

El PDU cambia de nombre dependiendo de la capa y se añade uno por cada para por
la que pasa.
\vspace{1em}

\textbf{PDU en Aplicación}: Se llaman \textbf{datos}.
\vspace{1em}

\textbf{PDU en Transporte}: Se llaman \textbf{segmento o datagrama } dependiendo de si es
TCP o UDP respectivamente.
\vspace{1em}

\textbf{PDU en Internet}: Se llaman \textbf{paquete}. Direccionamiento está en
el encabezado de red.
\vspace{1em}

Acceso a la red (medio) se divide en 2:

\begin{itemize}
	\item
	Enlace de datos
	\item
	Física
\end{itemize}
\vspace{1em}

\textbf{PDU en Enlace de datos}: Se llaman \textbf{trama}. Dirección MAC, control
de flujo (trailar).
\vspace{1em}

\textbf{PDU en Física}: Se llaman \textbf{bits}.
\vspace{1em}

Entonces:

PDUs = Datos -> segmento/datagrama -> Paquete -> Trama -> Bits.

\section{TCP y UDP}

\textbf{TCP}: Protocolo de Control de Tranferencia (Transmission Control Protocol).
\vspace{1em}

\textbf{UCP}: Protocolo de datagramas.
\vspace{1em}

\textbf{Datagrama}: Diagramas de usuario.
\vspace{1em}

\textbf{Características de TCP}
\vspace{1em}

Es seguro y es \textbf{orientado a la conexión}. Símil con una llamada.
\vspace{1em}

\textbf{Características de UCP}
\vspace{1em}

Es \textbf{no orientado a la conexión}. Símil con telegrama.
\vspace{1em}

\textbf{IP}.- Protocolo de direccionamiento (Internet Protocol). Solo entrega
paquetes.
\vspace{1em}

\textbf{Router}.- Se encarga de definir las rutas de envío de paquetes.
\vspace{1em}

\section{Modelo OSI y Modelos TCP/IP}

\textbf{Capa de presentación}.- Es el formato de los datos. Se encarga de que los
usuarios puedan enviar y recibir el mismo formato.
\vspace{1em}

\begin{itemize}
	\item
	Compresión de los datos
	\item
	Cifrado (si lo requiere)
\end{itemize}
\vspace{1em}

\textbf{Capa de sesión}.- Modos de diágologo.
\vspace{1em}

\textbf{Capa de Aplicación}.- Genera los mensajes.
\vspace{1em}

\section{Servicios}

\subsection{Servidor DNS}

Asocia una dirección IP con un  nombre de dominio.

\subsection{Servidor TELNET}

Escritorio remoto.
\vspace{1em}

Conexión remota para manipular la red desde otra ubicación.
\vspace{1em}

Sirve para el monitereo o manipulación de equipos a distancia.
\vspace{1em}

Desventaja: Es inseguro porque los datos van en texto plano.
\vspace{1em}

\subsection{Servidor Email}

SMTP, POP e IMAP(a veces).
\vspace{1em}

\textbf{SMTP}.- Simple Mail Transfer Protocol.
\vspace{1em}

\textbf{POP}.- Post Office Protocol.
\vspace{1em}

Para enviar y recibir correos.
\vspace{1em}

\subsection{Servidor DHCP}

Para asignar direcciones IP de manera dinámica.
\vspace{1em}

Tengo una dirección mientras estás navegando y se libera dicha dirección cuando
me desconecto.
\vspace{1em}

\subsection{Servidor web}

\textbf{HTTP}.- Hyper Text Transfer Protocol.
\vspace{1em}

Para cliente-servidor de páginas web.
\vspace{1em}

\subsection{Servidor FTP}

Para almacenar, transferir o descargar archivos de la capa de app.
\vspace{1em}

\textbf{Servicio o proceso}.- Es lo que está ejecutando un protocolo.
\vspace{1em}

\section{Modelo cliente-servidor}

El servidor procesa todos los requisitos el cliente.
\vspace{1em}

\textbf{Descarga}.- Servidor envía respuesta para que el cliente la procese.
\vspace{1em}

\textbf{Carga}.- El cliente solicita al servidor para que almacene info.
\vspace{1em}

EL servidor puede hacer procesamiento paralelo (centralizado).
\vspace{1em}

Los servidores son depósitos de info.
\vspace{1em}

\textbf{Daemon}.- Modo escucha (servicio que está activo).
\vspace{1em}

\section{DNS}

Es cliente-sevidor.
\vspace{1em}

\begin{enumerate}
	\item
	El servidor
	\item
	Petición (soliticat web page)
	\item
	Verificar
	\item
	Resuelve
	\item
	Devuelve la IP
	\item
	EL cliente recibe la IP
\end{enumerate}
\vspace{1em}

Hay jerarquías

\section{HTTP}

\begin{enumerate}
	\item
	Procesa la dirección IP
	\item
	Resuelve un código HTML
	\item
	Se obtiene la página web
\end{enumerate}

\section{HTTP}

\section{Control de flujo y conexión segura}

En la capa de transporte hay 2 protoclos: TCP y UDP.
\vspace{1em}

3 operacines básicas de confiabilidad:

\begin{itemize}
	\item
	Seguimiento de datos transmitidos
	\item
	Acuse de recibo de los datos recibidos
	\item
	Retransmisión de cualquier paquete sin acuse de recibo.
\end{itemize}
\vspace{1em}

\subsection{UDP}

\begin{itemize}
	\item
	No solicita reenvío
	\item
	Es rápido
	\item
	Tiene menor carga
	\item
	Entrega los datos cuando los recibe
\end{itemize}
\vspace{1em}

Ejemplos: Telefonía IP y streaming de video.

\subsection{TCP}

\begin{itemize}
	\item
	Confiable
	\item
	Acuse de recibo
	\item
	Reenvío de datos perdidos
	\item
	Los segmentos se entregan en el orden eviado
\end{itemize}
\vspace{1em}

Ejemplos: SMTP, POP y HTTP. Los correos se reciben o no, pero nunca a la mitad.

\section{Encabezado de Transporte}

Debe de llevar info para realizar el proceso de control.
\vspace{1em}

\textbf{Encabezado de segmento (TCP)}
\vspace{1em}

\begin{itemize}
	\item
	20 bytes
	\item
	Tiene muchos campos
	\item
	Mucha información
\end{itemize}
\vspace{1em}

\textbf{Encabezado de segmento (UDP)}
\vspace{1em}

\begin{itemize}
	\item
	8 bytes
	\item
	Tiene pocos campos
	\item
	Ligero
	\item
	Mínimo esfuerzo
\end{itemize}
\vspace{1em}

\section{Puertos}

Entrada y salida de datos.
\vspace{1em}

Los datagramas y segmentos deben indicar de qué puerto viene y hacia cual va.
\vspace{1em}

\textbf{Bien conocidos}.- 0-1023.

\textbf{Registrados}.- 1023-49151.

\textbf{Privados y/o dinámicos}.- 49151-65535.
\vspace{1em}

\textbf{Disclaimer}: La finalidad de este documento es servir como apoyo de estudio.
El autor de la versión original de este documento no se hace responsable del
uso indebido del mismo.

\end{document}
