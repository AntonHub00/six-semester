\documentclass{article}

%Configuraciones de idioma del documento
\usepackage[spanish]{babel}
%Configurar entrada {usar caracteres desde el teclado como á}
\usepackage[utf8]{inputenc}
%Fuente que pueda renderizar caracteres en nuestro idioma
\usepackage[T1]{fontenc}
%Fuente compatible
\usepackage{lmodern}

%\usepackage[top=2cm, left=2cm, right=2cm, bottom=2cm]{geometry}

\begin{document}

\title{Práctica 2 (Wix)}
\author{Antonio Emiko Ochoa Adame}
\maketitle

\section{Instalación}
No requiere de instalación ya que la selección de la plantilla,
la edición y la publicación se hace desde la misma página web.

\section{Configuración}
No se requiere configuración adicional, basta con registrarse en la
página para poder acceder a las funcionalidades de la misma.
Las únicas configuraciones necesarias son las de la página a
desarrollar, por ejemplo, idioma, divisa, método de pago, método de
envío, etc.

\section{Diseño}
La manipulación de las plantillas es muy básica pero intuitiva, se
puede hacer click derecho en la mayoría de elementos, editar, copiar,
pegar, cortar, cambiar de fuente, tamaño, orientación, etc.
Es ideal para aquellas personas que no tiene un conocimiento sobre
desarrollo web pero desean o necesitan un sitio web, aunque sea
básico.

\section{CMA}
El CMA está bien, es decir, no tiene un mayor problema en mostrar
la información necesaria al usuario; es aceptable.

\section{CDA}
En este lado no he interactuado mucho ya que no fue necesario para
esta práctica en específico, sin embargo, sé que es posible la
conexión de bases de datos y de otras herramientas para backend, la única desventaja de esto último es que, si no me equivoco, tienes
que pagar un precio específico para poder acceder a esas funcionalidades.

\section{Plugins}
Desconozco si es posible hacer uso de plugins. De cualquier manera, no
fue necesario para esta práctica.

\section{Costo y soporte}
Inicialmente, este servicio es gratuito, sin embargo, si se desea
obtener funcionalidades adicionales se tiene que pagar un precio
 a manera de renta, es decir, un cantidad cada mes para poder obtener
 ese beneficio.
 Si lo que se requiere es un sitio a manera de ``landing page'' o
 algún sitio web básico, es suficiente con la versión ``gratuita'', aunque
 a pesar de esto, pagar por los servicios adicionales no es caro en lo
 absoluto, es bastante barato considerando las ventajas que este ofrece.
\end{document}
