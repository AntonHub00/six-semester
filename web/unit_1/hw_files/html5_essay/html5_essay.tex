\documentclass{article}

%Configuraciones de idioma del documento
\usepackage[spanish]{babel}
%Configurar entrada {usar caracteres desde el teclado como á}
\usepackage[utf8]{inputenc}
%Fuente que pueda renderizar caracteres en nuestro idioma
\usepackage[T1]{fontenc}
%Fuente compatible
\usepackage{lmodern}

\usepackage[top=2cm, left=2cm, right=2cm, bottom=2cm]{geometry}

\begin{document}

\title{HTML 5}
\author{Antonio Emiko Ochoa Adame}
\maketitle

\tableofcontents
%\newpage

\section{Introducción}

\paragraph{}
En este documento se detallará brevemente la historia detrás del llamado HTML5
y lo que esto implica, ya que es un nuevo concepto visto por primera vez hace
aproximadamente diez años y que aún sigue vigente. También se hablará sobre la
confusión que llega a surgir cuando se intenta diferenciar HTML5 del HTML
tradicional, ya que desde que surgió ha estado inmerso en un debate entre las
organizaciones que apoyan un estándar para los desarrolladores de contenido web y los que
apoyan un estándar que vela por los desarrolladores de los navegadores como tal.
También se definirá brevemente los integrantes de HTML5 (CSS y Javascript) y el
rol que llevan dentro de esta definición, el porqué del uso de ellos y algunas
características de los mismos.

\section{Desarrollo}

\subsection{¿Qué es HTML5?}

\paragraph{}
HTML5, como su nombre lo indica, es la mayor y actual versión de HTML y esta
incluye a XHTML. Se dice que esta versión de está compuesta no solo por una
tecnología, sinó por un conjunto de tecnologías que definen las propiedades y
comportamientos de una aplicación web y esto se logra implementando un patrón
de ``marcado'' \cite{wiki_html5}.

\paragraph{}
Actualmente existe en dos formas estandarizadas: la \textbf{Recomendación HTML 5.2} por
la W3C\footnote{Wolrd Wide Web Consortium; una gran coalición de organizaciones.}
que está más dirida a los desarrolladores de contenido web y la
\textbf{HTML Living Standard} por la WHATWG\footnote{Un pequeño consorcio
compuesto por cuatro ``compañías'' que se dedican a desarrollar navegadores.},
dirigido principalmente a quienes se dedican a desarrollar navegadores
\cite{wiki_html5}.

\paragraph{}
Como se menciona en \cite{wiki_html5}, HTML5 como tal, es mostrado al público el
22 de Enero del 2008 y el objetivo de esta versión es mejorar el leguaje haciendo
posible el soporte para el contenido multimedia más actual y algunas otras
características nuevas. Todo esto se hizo con la intención de que el lenguaje
se mantuviera legible para humanos y que también fuese fácil para las computadoras
y dispositivos como navegadores web, traductores, entre otros, sin tener
la rigidez de XHTML y manteniendo una retrocompatibilidad con software más
antiguo.

\paragraph{}
HTML5 incluye modelos de procesamiento detallados que permite implementaciones
interoperables, mejora el marcado para documentos e inlcuye APIs para un desarrollo
web más complejo \cite{wiki_html5}. A continuación, algunas etiquetas añadidas:

\paragraph{}
Multimedia y gráficos:
\begin{itemize}
	\item <video>
	\item <canvas>
	\item <audio>
\end{itemize}

\paragraph{}
Elementos para estructura de páginas:
\begin{itemize}
	\item <main>
	\item <section>
	\item <footer>
	\item <header>
	\item <article>
	\item <aside>
	\item <nav>
	\item <figure>
\end{itemize}

\subsection{Diferencia entre HTML y HTML5}

\paragraph{}
Como se menciona en \cite{diff_1}, la diferencia en sí es nula, HTML5 sigue
siendo HTML, la diferencia es que HTML5 es una versión más de este, la cual
causó un gran impacto ya que se incorporaron nuevas etiquetas (como se mencionó
anteriormente) y nuevas APIs que mejoraron de manera significativa a la versión
anterior de este lenguaje.

\subsection{Integración de CSS y Javascript con HTML5}

\paragraph{}
Como ya se mencionó, HTML5 es conocido por estar integrado por varias
tecnologías y así contruir aplicaciones web más potentes e innovadoras;
CSS\footnote{Cascading Style Sheets, se utiliza para dar formato y presentación
a documentos HTML.} y
Javascript\footnote{Es un lenguaje interpretado de alto nivel usado para crear
sitios web dinámicos.}.

\paragraph{}
CSS describe el estilo de un documento HTML. Este lenguaje permite denifir la
forma en la que cada uno de los elementos deben se mostrados\cite{css}.

\paragraph{}
Javascript junto con CSS conforman el núcleo de lo que se conoce como HTML5.
Este es un lenguaje que permite crear páginas web interactivas, lo que lo hace
indispensable en la actualidad\cite{javascript}.
Aunque Javascript fue inicialmente creado para funcionar en el lado del cliente
(navegador) para poder mostrar contenido de manera dinámica y hasta asíncrona
, ahora ha sido llevado al lado del servidor ya que es considerado
un excelente lenguaje tanto en desempeño y funcionalidad como en sintaxis.

\section{Conclusiones}

\paragraph{}
HTML5 es un concepto un poco abstracto y hasta confuso en algunos casos; de
manera sencilla podría describir HTML5 como un compendio de tecnologías que
incluyen CSS, Javascript y un renovado HTML que, en conjunto, permiten
desarrollar tecnología web de una manera más potente y organizada que como se
solía hacer hace algún tiempo.

\paragraph{}
HTML5 surge primero como una necesidad, principalmente, sobre un mejor soporte
para el contenido multimedia, como pueden ser audio, video y gráficos en general,
ya que estos últimos años han tenido una gran popularidad y es actualemente
el mayor contenido de la web, desde mi punto de vista. Sin embargo, la tecnología
también se ve involucrada en un debate entre organizaciones que tratan de imponer
un estándar sobre esta tecnología, cada uno por su cuenta y en la forma en la
que ellos creen conveniente que debería ser utilizado. Esto genera disputas y
al final de cuentas solo son estándares que pueden ser sacrificados de vez en
cuando para lograr objetivos muy específicos cuando se habla de proyectos
web.

\paragraph{}
Mientras unos prefieren decir que es la versión 5 de este lenguaje de marcado
y otros pretender manter el lenguaje vivo sin una nomenclatura para las versiones,
siempre hay que considerar los cambios que realmente importan al momento de
desarrollar, y estos son los cambios y nuevas características agregadas, como
pueden ser la serie de etiquetas agregadas o las nuevas APIs disponibles para
el desarrollo de aplicaciones web más complejas y potentes, ya que implican una
nueva forma de programar web y que pueden facilitar las cosas cuando un proyecto
se torna demasiado grande.

\begin{thebibliography}{}
    \bibitem{wiki_html5}
	``HTML5''. [online]
	Available at: https://en.wikipedia.org/wiki/HTML5
	[Accessed 3 Feb. 2019].
    \bibitem{diff_1}
	L. Forgiarini. ``CUÁL ES LA DIFERENCIA ENTRE HTML Y HTML5?''.
	[online] Available at: https://luisforgiariniblog.com/cual-diferencia-entre-html-html5/
	[Accessed 3 Feb. 2019].
    \bibitem{css}
	``CSS Tutorial''.
	[online] Available at: https://www.w3schools.com/Css/
	[Accessed 3 Feb. 2019].
    \bibitem{javascript}
	``JavaScript''.
	[online] Available at: https://en.wikipedia.org/wiki/JavaScript
	[Accessed 3 Feb. 2019].
\end{thebibliography}

\end{document}
