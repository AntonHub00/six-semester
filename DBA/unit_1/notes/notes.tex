\documentclass{article}

%Configuraciones de idioma del documento
\usepackage[spanish]{babel}
%Configurar entrada {usar caracteres desde el teclado como á}
\usepackage[utf8]{inputenc}
%Fuente que pueda renderizar caracteres en nuestro idioma
\usepackage[T1]{fontenc}
%Fuente compatible
\usepackage{lmodern}

\usepackage[top=2cm, left=2cm, right=2cm, bottom=2cm]{geometry}

\begin{document}

\title{Unidad 1}
\author{Antonio Emiko Ochoa Adame}
\maketitle

\tableofcontents
%\newpage

\section{Administración de una DB}
\begin{itemize}
	\item
	Trooper habla más con el usuario.
	\item
	Skywalker soluciona problemas importantes (migraciones de datos, problemas más
	fuertes con la bases de datos).
	\item
	Vader sabre tratar a la gente, problemas con ingeniería social.
\end{itemize}

\paragraph{}
Las responsabilidad de DBA es facilitar el desarrollo y el uso de la base de datos.


\begin{itemize}
	\item
	Diseñar el modelo lógico y físico.
	\item
	Configuración del servidor
	\item
	Etc.
\end{itemize}

\subsection{Control de procesos repetitivos}

\paragraph{}
Automatizar tareas, por ejemplo, backups.

\textbf{Recomendación: Instalar SQLServer 2017 (versión developer tiene más funciones).}

\subsection{Administración de derechos y restricciones}

Qué se debe de hacer:
\begin{itemize}
	\item
	Ir a recursos humanos.
	\item
	Matriz de permisos.
\end{itemize}

1.- Se hace un análisis
2.- Se genera una tabla llamada ``matriz de permisos'', que sería hacer un
análisis de lo permisos, usuarios, cantidad de tablas en las bases de datos, etc.

\subsection{Procesamiento y memoria}

\paragraph{}
Se pueden asignar al DBMS la cantidad de recursos que puede utilizar (RAM y
ROM).
Se puede elegir la cantidad de procesadores para la DB para obtener un
rendimiento adecuado en la DB.

\paragraph{}
\textbf{NOTA}
\begin{itemize}
	\item
	Motor
	\item
	Servicio
	\item
	Instancia
\end{itemize}

\paragraph{}
\textbf{Motor}: Sería aqué que hace posible el servicio. Ayuda a hacer acciones en la DB.

\textbf{Servicio}: Permite abrir ventana. Intangible. Permite  o inicia al DBMS
y permite que el motor funcione con la intancia (gasolina de coche).

\textbf{Instancia}: ``Área de trabajo donde se tienen las bases de datos; espacio lógico''.
Organización de la base de datos.

\paragraph{}
Motor e intancia van ligados (moto-instancia).

Se pueden generar cualquier cantidad de intancias en cualquir DBMS.

En MYSQL se puede usar ``sysbench'' para observar cosas relacionadas con los
recursos.

\subsection{Desarrollo de la seguridad de la base de datos}
\paragraph{}

\subsection{Acceso a la base de datos y seguridad a nivel de aplicación}
\paragraph{}
Se tiene que ser conciente de cómo se lleva a cabo esta parte, incluso si
no es responsabilidad directa del DBA.
Se tiene que saber cómo funciona la seguridad desde el lado de la aplicación

\subsection{Administración del SGDB (DBMS)}
\begin{itemize}
	\item
	Pago del licenciamiento
	\item
	Documento de proceso de instalación, desinstalación y actualización.
	\item
	Verificación de parches y actualizaciones.
	\item
	Documentación de ruta de instalación, servidores, direcciones IP, etc.
	\item
	Permisos de acceso a servidores de producción y pruebas.
\end{itemize}

\subsection{Mantenimiento del repositorio}
\begin{itemize}
	\item
	Definición inicial de rutas de archivos primarios, secundarios, tablespaces,
	logs, unidades de disco.

	\item
	Monitoreo constante del espacio y crecimiento y definir un porcentaje cuando
	se deba levantar un alerta.

	\item
	Estadíticas mensuales.

	\item
	Documentar procesos de attach y dettach y de migración.
\end{itemize}

\textbf{Se hace Shrinking}

\paragraph{}
Se cuentan objetos en la bases de datos (tablas, campos, tipo de datos, usuarios,
etc.).
Observar el crecimiento de la DB, obtener gráfica de crecimiento.
Calcular la cantidad de tiempo que podría tardar en llenarse el almacenamiento.

\paragraph{}
Si la DB está en uso, se debe terminar de usar en la DB (voluntariamente o
`` a la fuerza'')

\textbf{dbcc} ayuda a darle mantenimiento a la DB.

\subsection{Monitoreo}

\paragraph{}
\begin{itemize}
	\item
	Seguridad
	\item
	Queries en proceso
	\item
	Indexación
\end{itemize}

\paragraph{}
Seguridad:
Agregar seguridad para que no hayan problemas con usuarios dentro de la misma
organización o accesosno autorizados.

\paragraph{}
Queries en proceso:
Se puede detectar si alguien está haciendo queries indebidos o queries no
optimizados o ``malos''.

\paragraph{}
Indexación:

\subsection{Mantenimiento}

\paragraph{}
\begin{itemize}
	\item
	SHRINK LOG
	\item
	DBCC SHRINKDATABASE;
	\item
	DBCC SHRINKDATABASE (DB, TRUNCATEONLY);
	\item
	DBCC CHECKTABLE
\end{itemize}


\paragraph{}
Integridad:

La manera en la que está organizada la DB físicamente (en el disco).

\textbf{Bloque} 8 KB (un porción de una tabla).

Se necesitan varios \textbf{bloques} para guardar varias \textbf{tablas}.

Un conjunto de \textbf{bloques} se llama \textbf{Extensión}.

Varias \textbf{extenciones} forman un \textbf{segmento}.

DBCC CHECKTABLE ayuda a checar la integridad.

\paragraph{}
Indices:

Son archivos.

Se deben reoganizar para optimizar o que funcionen correctamente.

\paragraph{}
Limpieza de logs binarios MySQL.

Optimización de tablas MySQL (OPTIMIZE TABLE).

Vaciamiento de caché de consultas.(FLUSH QUERY CACHE).

Rotación de logs binarios MySQL. (Se vacían manualmente si ya no hay espacio).
(Puede que tenga que ser creada una nueva bitácora)

\subsection{(11)Auditorías de logs}
Se pueden seleccionar que campos se desean auditar.

\subsection{(12)Respaldo y restauración}

RPO: Cuando se dan señales de que algo anda mal.

Desastre: (En el mejor escenario hay un respaldo).

RTO: Cuando se concluye que todo volverá a funcionar normalmente.


\paragraph{}
\begin{itemize}
	\item
	Frecuencia
	\item
	Horario
	\item
	Rotación
	\item
	Bitácora
	\item
	Tareas automatizadas
	\item
	Caja fuerte
	\item
	Sitio alterno
\end{itemize}

\subsection{(13)Concurrencia}

\begin{itemize}
	\item
	Manejo de transacciones
	\item
	Llaves primarias
	\item
	Índices
	\item
	Propiedades ACID:
		\begin{itemize}
		\item
		Atomic
		\item
		Consistent
		\item
		Isolated
		\item
		Durable
		\end{itemize}
\end{itemize}

\textbf{Manejo de transacciones}
\vspace{1em}

Cuando hay un select se genere un ``bloqueo compartido''.
Un select se atiende más rápido porque solo es una lectura (bloqueo compartido).

Cuando hay un update se genere un ``bloqueo exclusivo''.

Begin transaction, rollback y commit ayudan para asegurar la consistencia de la
información.
\vspace{1em}

\textbf{Llaves primarias}

Evita redundancia de tuplas
\vspace{1em}

\textbf{Índices}

Ayuda a que los queries se ejecuten más rápido.
\vspace{1em}

\textbf{ACID}

Deben asegurar que cuando se haga una transacción se debe asegurar ACID.

Definir ACID.

Durabilidad.- Si es exitosa la transacción, debe hacerse un commit para que los
cambios permanezcan o ``persistan''.

\subsection{(14)Bloqueos}

Exlusivo, compartido, deadlock.

Se bloquea cuando dos usuarios quieren utilizar el mismo recurso al mismo
tiempo. A esto se le llama \textbf{deadlock}.

Se debe monitorear para ver si ocurren deadlocks.

Se debe liberar el deadlock manualmente, pero esto puede generar inscosistencia
en la información.

\subsection{(15)Problemas de pérdida de información}

\begin{itemize}
	\item
	Correciones manuales automatizadas.
	\item
	Correcciones masivas.
\end{itemize}

\subsection{(16)Migraciones}

\begin{itemize}
	\item
	De tabla a tabla.
	\item
	De una DB a otra DB.
	\item
	A una DB de producción.
	\item
	A una DB de consultas.
	\item
	A una DB de pruebas.
	\item
	DB distribuída.
	\item
	Implicaciones en llaves primarias y autoincrementables.
\end{itemize}
\vspace{1em}

Cuando un campos se usa mucho se le pude generar un índice para optimiziar
una búsqueda.
\vspace{1em}

Si no es suficiente, se puede partir la tabla para en campos importantes
para poder agilizar los queries.
\vspace{1em}

Una tabla pequeña puede creecer demasiado sin darnos cuentas, por es
es importante sacar estadísticas, así se puede obsevar el comportamiento
de la DB y poder tomar desiciones en base a dicha información.
\vspace{1em}

\section{Respaldos}

Técnicas de respaldo.
\vspace{1em}

Inicio de código

USE MASTER;

BACKUP DB-NAME

Fin de código
\vspace{1em}

\textbf{Página o bloque}.- Unidad mínima de almacenamiento de una DB.

Varios bloques/páginas forman un extent se guardan tablas.

Varios extent forma un segmento.
\vspace{1em}

Una DB siempre está divido en 2 archivos: DATA y LOG.
\vspace{1em}

Automatización de los respaldos es lo más adecuado.
\vspace{1em}

Faltan temas: bak en MySQL.

Práctica: Se generará código a partir de un modelo relacionar, generar, respaldos,
procedures, trigger, tareas de mantenimiento y monitoreo, insetar tuplas, etc.

\section{Permisos}

\begin{itemize}
	\item
	Grant (Da permisos)
	\item
	Deny (Niega ciertos permisos)
	\item
	Revoke (Revoca permisos, quita los permisos que tenga)
\end{itemize}
\vspace{1em}

\textbf{Procesos de negocio}
\vspace{1em}

Conocer el proceso de negocio permite saber qué permisos necesita cada tipo de
usuario.

Tener claro quienes son mis usuarios y cuáles tablas hay.
\vspace{1em}

Es importante saber quién está conectado en la DB.

EL comando EXEC SP\_WHO permite saber esto.

SELECT * FROM SYSPROCESSES WHERE SPID = 55;

SELECT * FROM SYSOBJECTS WHERE XTYPE = 'U';

Introducir aquí query que muestra queries corridos (desencriptados.).
\vspace{1em}

Falta: Funciones para crear índice, funciones de mantenimiento y función de
importar exportar.

\vspace{1em}
Práctica 1 será para hacer en una semana.

\vspace{1em}
\textbf{INDICES}

\vspace{1em}
Símil con índice de libro; referencia a una localidad específica.

Una tupla puede estar guardada en n páginas.

\vspace{1em}
\textbf{Indice}.- Se crea una tabla o parte de ella y se ordena por el campo seleccionado.

Es como un resumen de la información de la tabla, con el fin de acceder
rápidamente a lo que más se suele utilizar.

\vspace{1em}
Existen 2 tipos principales de índices:

\vspace{1em}
1.-Uno solo tiene una referencia (el visto anterior) (Non-clúster en MySQL Server)

2.-Otro tiene un puntero que dice en dónde se encuentra la información.
(Clúster en MySQL Server).

\vspace{1em}
En MySQL:

\begin{itemize}
	\item
	Index
	\item
	Primary (Cuando no quiere que se repita la info en una columno)
	\item
	Fulltext (Los que tienen muchos bytes de texto, conviene usar este)
	\item
	Spatial (Campos en los que se guardan coordenadas)
\end{itemize}



\end{document}
