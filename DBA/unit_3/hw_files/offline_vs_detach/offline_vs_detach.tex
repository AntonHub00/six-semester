\documentclass{article}

%Configuraciones de idioma del documento
\usepackage[spanish]{babel}
%Configurar entrada {usar caracteres desde el teclado como á}
\usepackage[utf8]{inputenc}
%Fuente que pueda renderizar caracteres en nuestro idioma
\usepackage[T1]{fontenc}
%Fuente compatible
\usepackage{lmodern}

%\usepackage[top=2cm, left=2cm, right=2cm, bottom=2cm]{geometry}

\begin{document}

\title{Offline vs Detach (SQL Server)}
\author{Antonio Emiko Ochoa Adame\\
Julia Lizbeth Huerta Álvarez\\
José Ángel Cortés Gómez}
\maketitle

Tanto ``offline'' como ``detach'' hacen que la base de datos quede inaccesible
para lo usuarios, sin embargo, cuando se realiza detach, los metadatos de la
base de datos a la que se le aplicó el comando quedan eliminados, es decir, la
información acerca del archivo de la base de datos y otros datos que se
pueden observar en la vista ``sys.database''. Al hacer ``offline'', todos los datos
relacionados con la base de datos se mantienen y se pueden observar en las vistas
de sistema de SQL Server \cite{sqlservergeeks}.

\vspace{1em}

Si se pone offline la base de datos, esta se muestra en la interfaz gráfica del
SSMS como ``(offline)'' haciendo referencia al estado en el que se encuentra.
Otra diferencia es que, cuando se utiliza detach, primero eliminan las conexiones
ya que la conexión se pone en modo ``single user''. Al realizar lo anterior, si
se intenta hacer un query a sys.database no devolverá nada \cite{sqlservergeeks}.

\vspace{1em}

Cuando se hace detach, se elimina el registro de la base de datos dentro de SQL
Server, es decir, el manejador deja de reconocerla. Para poder volver a usar la
base de datos se tiene que indicar donde se encuentra el archivo.

Cuando se pone offline, el registro de la base de datos se mantiene intacto,
por lo que si se desea volverla a usar, basta con ponerla online de nuevo \cite{microsoft}.

\vspace{1em}

Detaching/attaching se utiliza para mover los archivos de base de datos a otros
servidores o a otra localidad; offline no permite mover archivos. Al realizar
detaching/attaching a la base de datos, esta se elimina del manejador (SQL Server), pero
los archivos data y log se mantendrán intactos \cite{microsoft}.

\begin{thebibliography}{}
    \bibitem{sqlservergeeks}
	SQL Geek. ``Detach or take offline in SQL Server''.
	[online] Available at: https://www.sqlservergeeks.com/detach-or-take-offline-in-sql-server/
	[Accessed 26 Mar. 2019].
    \bibitem{microsoft}
	``whats the difference between detaching and taking a db offline?''.
	[online] Available at: https://social.msdn.microsoft.com/Forums/sqlserver/en-US/693123ec-9dde-496d-b9c9-daea5eef3991/whats-the-difference-between-detaching-and-taking-a-db-offline
	[Accessed 26 Mar. 2019].
\end{thebibliography}

\end{document}
