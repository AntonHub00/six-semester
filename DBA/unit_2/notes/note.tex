\documentclass{article}

%Configuraciones de idioma del documento
\usepackage[spanish]{babel}
%Configurar entrada {usar caracteres desde el teclado como á}
\usepackage[utf8]{inputenc}
%Fuente que pueda renderizar caracteres en nuestro idioma
\usepackage[T1]{fontenc}
%Fuente compatible
\usepackage{lmodern}

\usepackage[top=2cm, left=2cm, right=2cm, bottom=2cm]{geometry}

\begin{document}

\title{Unidad 2}
\author{Antonio Emiko Ochoa Adame}
\maketitle

\tableofcontents
%\newpage

\section{Componentes de la arquitectura de un SGBD}

Objetivos de un SGDB:

\begin{itemize}
	\item
		Independencia de los datos:
	\item
		Integridad de los datos:
	\item
		Seguridad de los datos:
\end{itemize}

\vspace{1em}
\textbf{Servicios de un SGDB}

Creación de un SGDB (aparición)

Antes se guardaba la info en archivos sobre un OS.

Conjunto de programas que definían y trabajaban sus propios datos (no funcionó;
había que estar reparando a cada rato)

Cuando se dió la concurrencia la información se corrompía.

Se necesitó, entonces, un sistema que lidiara con todos estos problemas.

\vspace{1em}
\textbf{Bases de datos plana (una tabla)}

Se parecía a Excel, pero era insegura.

\begin{itemize}
	\item
		Había redundacia e inconsistencia.

	\item
		Dificultad para acceso a datos.

	\item
		Separación y aislamiento de los datos.

	\item
		Porblemas de seguridad de datos.

\end{itemize}

\section{Niveles}

\subsection{Nivel interno físico}

El más cercado al almacenamiento físico, es decir, como está ordenado en la compu.

Describe la estructura física de la DB mediante el esquema interno.

Especifica un modelo físico y describe detalles de cómo se almacenan físicamente
los datos; archivos, organización, métodos de acceso, longitud, campos, etc.

\subsection{Nivel externo o ...}
Memoria externa de la compu.
Cómo vemos los datos y las tablas desde afuera.

\subsection{Nivel conceptural}

Modelo ER, modelo jerárquico, atributos, operaciones entre los usuarios y
restricciones.

Representa la info contenida en la DB.

Visión desde un punto de vista organizativa. El porqué de los modelos, ER,
``el cómos realacionan los datos''.

\subsection{Nivel lógico}

Es un SGDB concreto, es el esquema lógico de la representación de entidades y
relaciones.

\vspace{1em}
El SGDB debe tranformar cualquier petición del usuario.

Actualmente son mejores: se puede crear índices, monitorear, auditar, implementar
seguridad, concurrencia de miles o millones de usuarios.

\section{Componentes de los SGDBs}

\subsection{Lenguajes}

\begin{itemize}
	\item
		DDL: Leguaje de defición de datos.
	\item
		DML: Leguaje de manipulación de datos.
\end{itemize}

Diccionario de datos: Información sobre la defición de los datos (SON LOS
METADATOS).

Proporciona info importantes acerca tablas, usarios, etc.

\subsection{Seguridad e integridad}

Restricción de acceso de usuarios.

Mecanismo creado para garantizar la seguridad.

Que no acceda cualquier persona, que los datos estén seguros.

\subsection{Requerimiento para la instalación de un SGDB}

\textbf{Ediciones de SQL Server}

\begin{itemize}
	\item
		Enterprise (clúster)
	\item
		Standard
	\item
		Developer (www.dreamspark.com)
	\item
		Express
\end{itemize}

\vspace{1em}
\textbf{Idioma - Collation}

Español.

\vspace{1em}
\textbf{Collation}.- Nos dice qué símbolos están permitidos usar en la DB.

\vspace{1em}
\textbf{Requerimientos}

\begin{itemize}
	\item
		.net framework 3.0
	\item
		IIS (para apps web)
	\item
		Visual Studio (para algunas funciones si se usa Microsft Office)
\end{itemize}

\vspace{1em}
\textbf{Asistente}

\begin{itemize}
	\item
		Aplicaciones/herramientas
	\item
		x servicios de consultas
	\item
		Polybase
	\item
		Analisis Services
	\item
		Polybase
	\item
		Autenticación mixta
\end{itemize}

%Line to separate
\noindent\makebox[\linewidth]{\rule{\paperwidth}{2.0pt}}

\vspace{1em}
Oracle es muy robusto.

\section{Arquitecturas de un SGBD}

\begin{tabular}{ c | c | c | c |}
   Similitudes & MySQL & SQL Sever \\ \\
	Transacciones & Todos es posible el uso de transacciones\\
	Vistas & Si & No &\\
	Plataformas & Linux, Windows &  &\\
 \end{tabular}

\subsection{Arquitectura de SQL Server}
\begin{itemize}
	\item
		Servicios Broker
	\item
		Réplica
	\item
\end{itemize}

\subsection{Arquitectura de MySQL}
Arquitectura Lógica y física.

\subsection{Arquitectura de Postrgess}


%Line to separate
\noindent\makebox[\linewidth]{\rule{\paperwidth}{2.0pt}}

\section{MySQL}

Basado en transacciones.

Enterprice Edition tiene moitoreo.

Se paga (y conviene) porque se tiene soporto técnico.

MySQL Cluster CGE

Para distriuir en distintas localidades físicas.

MySQL Community Edition.

Hecha por la comunidad

Bases de datos relacionales y no relacionales.

No soporte (solo por la comunidad).

Muy parecido a la versión Standard

\vspace{1em}
\textbf{Replicación}

Mejor rendimiento y velocidad porque la replicación puede obtener recursos de
varios lados al mismo tiempo.

Si falla un equipo, se tiene aún lo demás conectados.

\vspace{1em}
\textbf{Instalación}

Developer

\vspace{1em}
\textbf{Requerimientos}

RAM

HD

Memoria Virtual

Tamaño máximo de DBs

SOs

\section{Administración del espacio en disco}

\subsection{Definición de conceptos}

\begin{itemize}
	\item
		Tabla
	\item
		Bloque (página)
	\item
		Extensión
	\item
		Segmento
\end{itemize}

\textbf{Bloque} o página es la unidad más pequeña (múltiplo del tamaño de las
páginas del OS) de almacenamiento.

Es mejor que los bloque sean del mismo tamaño que los del OS o múltiplo porque
así no se sobrecarga el procesador haciendo operaciones innecesarias.

Un registro iría en un bloque (hablando de espacio físico).

\textbf{Extensión}: Cuando se configura el tamaño de un elemento se usan
extensiones para poder aumentar el tamaño de dicho objeto.

\textbf{Segmento}: grupos de extensiones que forman lo objetos de la DB.

\textbf{Datafile}: (*.mdf ó *.ndf) .Solo puede pertenecer a un determinado
tablespace.

\end{document}
