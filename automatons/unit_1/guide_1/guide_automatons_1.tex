\documentclass{article}

%Configuraciones de idioma del documento
\usepackage[spanish]{babel}
%Configurar entrada {usar caracteres desde el teclado como á}
\usepackage[utf8]{inputenc}
%Fuente que pueda renderizar caracteres en nuestro idioma
\usepackage[T1]{fontenc}
%Fuente compatible
\usepackage{lmodern}

\usepackage{alltt}

\usepackage{amsmath}

\begin{document}

\title{Guía de Lenguajes y Autómatas Unidad 1}
\author{Antonio Emiko Ochoa Adame}
\maketitle

\section{Definiciones}

\subsection{Conjunto}

Es una \textbf{colección} bien definida de objetos, a los cuales
se les llama \textbf{elementos}.
\vspace{1em}

$a \in A $ : el elemento a pertencece al conjunto A.

$a \notin A $ : el elemento a no pertencece al conjunto A.

\paragraph{}
Las llaves ``$\{\}$'' para describir un conjunto.

\subsection{Conjunto universal U}

Todos los posibles objetos que se consideran para una determinada clase de
conjuntos.

\subsection{Cardinalidad}

Cantidad de elementos que existen en un conjunto.

\textbf{NOTA}: Si un cojunto es infinito,  \textbf{no} tiene cardinalidad.

Hay infinitos \textbf{enumerables}.

(Dos conjuntos tienen la misma cardinalidad) ? Son equivalentes.

\subsection{Conjunto vacío}

No contiene elementos.

$\o = \{\}$

Carnidalidad = 0

\subsection{Conjuntos de números}

\subsubsection{Naturales (N)}

Se les llama así porque resulta del proceso natural de contar.

1, 2, 3, 4, 5, ...

\subsubsection{Enteros (Z)}

Positivos y negativos.

\subsubsection{Reales (R)}

Decimales, infinitesimales, etc.

\subsection{Aleph}

\textbf{Aleph 0}:
Es la cardinalidad del conjunto de todos los números naturales.
\textbf{Infinitos enumerables.}
\vspace{1em}

Ejemplos:
\begin{itemize}
	\item
	Los cuadrados de los números
	\item
	Potencias perfectas
	\item
	Números primos
	\item
	Números pares
\end{itemize}

\textbf{Aleph 1}: Es la cardinalidad del conjunto de todos los números
ordinales contables. \textbf{Infinitos no enumerables.}
\vspace{1em}

Ejemplos:
\vspace{1em}

No defined.

\subsection{Operaciones con conjuntos}

\subsubsection{Intersección}

Sean A y B, (A$\cap$B), todos los elementos de A que también estań en B.
\vspace{1em}

Si una intersección resulta vacío, se dice que es un conjunto ``disjunto''.

\subsubsection{Unión}

A$\cup$B. Conjunto formado por todos los elementos de de A o B.

\subsubsection{Complemento}

$A^c$. Elementos del universo que no pertenece a A.

\subsubsection{Diferencia}

A$-$B. Todos los elementos de A que no están en B.
A$-$B $= A\cap B^c$

\subsubsection{Diferencia simétrica}
Todos los elementos de A que no están en B y todos los elementos de B que no
están en A.
\vspace{1em}

A$\oplus$B $=$ (A$-$B) $\cup$ (B$-$A) $=$ (A$\cup$B) $-$ (A$\cap$B)

\subsubsection{Igualdad de conjuntos}

Dos conjuntos son iguales cuando los elementos son los mismos en ambos conjuntos.

\subsubsection{Subconjunto}

A$\subset$B. Si cada elemento que pertenece a A también es un elemento de B.
\vspace{1em}

A$\subset$B. \textbf{Propio} (No deben ser iguales).

A$\subseteq$B. \textbf{Impropio} (Pueden ser iguales).

\subsubsection{Cardinalidad de la unión de dos conjuntos}

N(A$\cup$B) = N(A) $+$ N(B) $-$ N(A$\cap$B)

\subsubsection{Potencia}

Conjunto formado por todos los posibles subconjuntos $2^A$.

\subsubsection{Partición}

Todos los elementos de S deben estar en alguna partición.

Un elemento solo puede estar en un conjunto.
\vspace{1em}

A$_1\cap$A$_2$= S ó A$_1\cap$A$_2$ = $\o$

\subsubsection{Producto cartesiano}

A$\times$B como el conjunto de todas las parejas ordenadas (a, b) tales que
a$\in$A y b$\in$B.

\textbf{NO ES CONMUTATIVA}: (2, 1) $\neq$ (1, 2).

\textbf{NOTA}: El resultado son tuplas.

\subsection{Relaciones}

\subsubsection{Relación}

Subconjunto de un producto cartesiano.

A$\rightarrow$B

A: Dominio (Proviene del primer conjunto).
B: Contradomio (Proviene del segundo conjunto).

\subsubsection{Relación reflexiva}

En la matriz de adyacencia si solo hay ``unos'' en la diagonal.

Ejemplos: (1, 1), (2, 2), (3, 3).

\subsubsection{Relación irreflexiva}

En la matriz de adyacencia si solo hay ``ceros'' en la diagonal.

\subsubsection{Relación simétrica}
Si (a, b) $\in$ R y (b, a) $\in$ R.

\subsubsection{Relación asimétrica}
No puede haber elementos del tipo (2, 2), pero también es irreflexiva.

\subsubsection{Relación antisimétrica}
Una relación binaria R sobre un conjunto A es antisimétrica cuando se da que
si dos elementos de A se relacionan entre sí mediante R, entonces estos elementos son iguales.

\subsubsection{Relación transitiva}
(a, b) $\in$ R y (b, c) $\in$ R, entonces (a, c) $\in$ R.

\subsubsection{Relación de equivalencia}

Es una realación equivalencia si es reflexiva, simétrica y transitiva.

Define una partición.

\subsection{Símbolo}

Es la representación tangible de un concepto.

\begin{itemize}
	\item
	\textbf{Letra}: Símbolo gráfico (representa un sonido concreto).
	\item
	Un dígito es la representación de un valor numérico.
	\item
	Símbolos de ciencias e ingeniería.
	\item
	Señales
	\item
	Emblemas y logotipos.
\end{itemize}

Hay símbolos que pueden ser perceptibles por otros sentidos.

\subsection{Alfabeto}

Conjunto finito no vacío de símbolos.

Pueden tener orden, pero no todos.

Se representa con $\Sigma$.
\vspace{1em}

Ejemplos:

\begin{itemize}
	\item
	$\Sigma = \{0, 1\}$ (Binario)
	\item
	$\Sigma = \{\alpha, \beta, \gamma, \delta, ..., \omega\}$
	\item
	$\Sigma = \{READ, INPUT, GET, FOR, ..., IF\}$
\end{itemize}

\subsubsection{Propiedades de los alfabetos}

Se pueden usar las siguientes operaciones: $\cup, \cap, -, \oplus$.

$\Sigma_1 \cup \Sigma_2$ es un alfabeto.

$\Sigma_1 \cap \Sigma_2$ es un alfabeto si $\Sigma_1$ y $\Sigma_2$ no son
disjuntos.

$\Sigma_1 - \Sigma_2$ es un alfabeto si $\Sigma_1 \not\subseteq \Sigma_2$.

$\Sigma_2 - \Sigma_1$ es un alfabeto si $\Sigma_2 \not\subseteq \Sigma_1$.

$\Sigma_1 \oplus \Sigma_2$ es un alfabeto si $\Sigma_1 \ne \Sigma_2$.
\vspace{1em}

Ejemplos:
\vspace{1em}

Dado:

$\Sigma_1 = \{0, 1, 2, 3, 4\}$
$\Sigma_2 = \{0, 2, 5, 6\}$
$\Sigma_3 = \{0, 2\}$
\vspace{1em}

Realizar las siguientes operaciones:

$\Sigma_1 \cup \Sigma_2 = \{0, 1, 2, 3, 4, 5, 6\}$

$\Sigma_1 \cap \Sigma_3 = \{0, 2\}$

$\Sigma_2 - \Sigma_3 = \{5, 6\}$

$\Sigma_3 - \Sigma_1 =$ No existe alfabeto. El alfabeto $\o$ no es posible.

$\Sigma_1 \oplus \Sigma_3 = \{1, 3, 4\}$

\subsection{Cadena}

Secuencia finita de símbolos de un alfabeto dado, yuxtapuestos uno a continuación
de otro.
\vspace{1em}

Ejemplos:
\vspace{1em}

$\Sigma = \{0, 1\}$

$x = 000011111, y = 10101, w = 11, z = 0$
\vspace{1em}

$\Sigma = \{a, b, c, d, e\}$

$x1 = beca$

$x2 = aaaaaaaaaa$

$x3 = c$

$x4 = \varepsilon$

$x5 = decada$
\vspace{1em}

\textbf{NOTA}: El orden importa, es decir, $luis \ne suli$

\subsection{Cadena vacía}

Se forma por una secuencia de cero símbolos de cualquier alfabeto.

La cadena vacía se denot con $\varepsilon$.

La longitud de una cadena es la cantidad de símbolos que contiene.

\subsection{Ambigüedad de cadenas}

Se suele usar \textbackslash 0 (o \textbackslash O?) para eliminar ambigüedad de cadenas vacías.
\vspace{1em}

Ejemplo de ambigüedad de cadenas:
\vspace{1em}

$\Sigma = \{01, 1, 0\}$

$x = 0101$ No se sabe que longitud tiene.

\subsection{Longitud de una cadena}

$w_1 = 101011$, entonces $|w_1| = 6$

$w_2 = 10110100101$, entonces $|w_2| = 11$

$w_3 = \varepsilon$, entonces $|w_3| = 0$

\subsection{Concatenación}

Es yuxtaponer dos cadenas, una a continuación de la otra.

$x \cdot w$ o $xw$

\subsection{Propiedades de la concatenación}

\textbf{NOTA}: No es conmutativa, es decir, $wx \ne xw$.
\vspace{1em}

La longitud de la concatenación es igual a la suma de las longitudes de las
cadenas individuales.
\vspace{1em}

$|wx| = |w| + |x|$

$|wx| = |xw|$ son diferentes, pero tienen la misma longitud.
\vspace{1em}

La cadena vacía es el \textbf{elemento neutro} de la concatenación.
\vspace{1em}

Sea $w$ un cadena, para cualquier $n \ge 0$, se tiene la enésima potencia
de $w$.

\begin{equation*}
    w^n = \begin{cases}
    \varepsilon & n = 0\\
    ww^{n-1} & n > 0\\
    \end{cases}
\end{equation*}

Ejemplos:
\vspace{1em}

Con $w = abc$
\vspace{1em}

$w^0 = \varepsilon$

$w^1 = ww^0= abc \varepsilon = abc$

$w^2 = ww^1 = abcabc$
\vspace{1em}

\textbf{NOTA}: La potencia \textbf{no} es conmutativa.

\paragraph{}
\textbf{NOTA}: El autor de este documento no se hace responsable de uso indebido del
mismo.

\end{document}
