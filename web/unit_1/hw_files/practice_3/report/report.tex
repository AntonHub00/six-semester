\documentclass{article}

%Configuraciones de idioma del documento
\usepackage[spanish]{babel}
%Configurar entrada {usar caracteres desde el teclado como á}
\usepackage[utf8]{inputenc}
%Fuente que pueda renderizar caracteres en nuestro idioma
\usepackage[T1]{fontenc}
%Fuente compatible
\usepackage{lmodern}

\usepackage{graphicx}

%\usepackage[top=2cm, left=2cm, right=2cm, bottom=2cm]{geometry}

\begin{document}

\title{Práctica 3 (HTML)}
\author{Antonio Emiko Ochoa Adame}
\maketitle

\section{Conclusiones}

\paragraph{}
El hecho de que en HTML5 se hayan implementado nuevas etiquetas que definen
la semántica o significado de los elementos dentro de un documento es muy
conveniente. Con estas etiquetas podremos decirle a los motores de búsqueda de
una manera más fácil y precisa en dónde puede encontrar el contenido de
relevancia para los usuarios.

\paragraph{}
Las etiquetas trabajadas también hacen posible que los usuarios puedan
indentificar de una manera acertada el contenido que están buscando y que
así obtengan una mejor sugerencia con el motor de búsqueda de preferencia.

\section{Evidencia de los archivos HTML}

\paragraph{}
\begin{figure}
	\includegraphics[width=\linewidth]{pra_3.png}
	\caption{index.html}
	\label{index}
\end{figure}

\paragraph{}
\begin{figure}
	\includegraphics[width=\linewidth]{pra_3_1.png}
	\caption{Tema\_1\_1.html}
	\label{tema_1}
\end{figure}

\end{document}
