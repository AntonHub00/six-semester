\documentclass{article}

%Configuraciones de idioma del documento
\usepackage[spanish]{babel}
%Configurar entrada {usar caracteres desde el teclado como á}
\usepackage[utf8]{inputenc}
%Fuente que pueda renderizar caracteres en nuestro idioma
\usepackage[T1]{fontenc}
%Fuente compatible
\usepackage{lmodern}

\usepackage[top=2cm, left=2cm, right=2cm, bottom=2cm]{geometry}

\begin{document}

\title{Evolución de las aplicaciones web}
\author{Antonio Emiko Ochoa Adame}
\maketitle

\tableofcontents
%\newpage

\section{Introducción}
En este documento se hablará sobre las aplicaciones web, su historia y de su
importancia en la vida de las personas. También se cubrirán aspectos sobre
la web desde su pasado, la actualidad y hasta el futuro de la misma.
La web no es algo que se haya inventado de la noche a la mañana, es algo que
ha estado en constante desarrollo desde su invención y a pesar de que es tecnología que
apareció mucho tiempo atrás, los avances tecnológicos han sido grandes y las
limitaciones técnicas han ido desapareciendo con el paso de los años.
Se hablará también sobre los navegadores que existieron (y algunos que siguen
existiendo) mucho tiempo atrás y sobre las intenciones de los mismos, ya que no
todos estaban dirigidos para el mismo tipo de público ni tenían los mismos
intereses.
También se comentarán algunos hechos históricos tanto de la web como de internet
en general que han sido aportaciones de importancia y que muy probablemente
fueron factores que han esculpido la web como la conocemos actualmente.

\section{Desarrollo}

\subsection{¿Qué es una aplicación web?}
Una aplicación web (también conocida como web app) es un programa computacional
que funciona sobre un ``paradigma'' cliente-servidor al igual que cualquier otro
de este tipo, sin embargo, lo que caracteriza a esta clase de software es que
estos programas corren en un navegador web\cite{wiki_web_app}.

Este tipo de aplicaciones funciona de la siguiente manera: Al tratarse de una
aplicación cliente-servidor, existe una comunicación entre el navegador y el
servidor; el navegador hace la función de interfaz de usuario (donde el usuario
puede interactuar con el software) y el servidor ser encargará de procesar los
datos y las peticiones hechas por los usuarios de estos servicios. Esto es así
por lo menos en los modelos cliente-servidor actuales.

Como se menciona en \cite{um}, el término ``Aplicación Web'' muchas veces es
confundido con ``Sitio Web'', o mejor dicho, la diferencia entre estos dos
se muestra al usuario de forma transparente. Como se menciona en el artículo,
un sitio web estático es una herramienta que es usada para comunicación en la
mayoría de los casos. Una aplicación web, por su parte, permite al usuario la
realización de una tarea en concreto, por ende, este software permite la
interacción con el usuario y con los datos del mismo, brindando así una forma
especializada de realizar aquella tarea específica antes mencionada.

\subsection{Historia}
La evolución de las aplicaciones web ha sido muy clara, ya que antes de estas y
como se menciona brevemente en la sección anterior, era común el uso de sitios
web estáticos. Estos funcionan igualmente en el nevegador, sin embargo, su función
consiste simplemente en mostrar texto; el texto no cambia, simplemente muestra
al usuario información acerca de un tema arbitrario, desde divulgación
científica hasta artículos de ocio. Ejemplos de esto podrían ser blogs personales
o de divulgación, enciclopedias online, landig pages, ventas de artículos
(en forma de catálogo, por ejemplio), etc.

En \cite{um} se menciona que Perl\footnote{Lenguaje de programación creado en
1987 por Larry Wall con características similares a sh, C, AWK, etc., y es
utilizado por su gran desempeño en el procesamiento de texto pero sin las
limitación de los lenguajes antes mencionados\cite{wiki_perl}.} fue uno de los primeros
lenguajes de programación que se utilizaron para realizar aplicaciones web.
Después de esto surgió un nuevo lenguaje de programación conocido como PHP,
gracias al cual fueron desarrolladas muchas tecnologías como Facebook, Google,
Wikipedia, etc.

Después de eso, Netscape (Uno de los navegadores más antiguos), anunció una
nueva tecnología llamada Javascript lo que supuso un cambio magistral en la web,
ya que permite cambiar de forma dinámica (y opcionalmente asíncrona) el
contenido de un sitio web, pasando así de estático a dinámico, haciendo así los
primeros aportes hacia el camino de las web apps.
Javascript fue y sigue siendo utilizado actualmente debido a que es uno de los
lenguajes más importantes y prometedores al momento de escribir este documento.
La razón por la que sigue siendo utilizado se debe a la forma en la que
funcionan las web apps, siendo casi indispensable en cualquier proyecto de este
tipo. Esto se refiere a que la forma en la que cambió la interacción entre el
usuario y la web,  hace que sea más natural e intuitiva, esperando un resultado
específico con dicha interacción.

En 1996 se lanzó Hotmail como un servicio de correo en línea para que cualquier
persona pudiese hacer uso de él.

En 1997 se lanzó Flash (Flash Shockwave) el cual permitió crear páginas web
más interactivas.

El año de 1998 se marcó como un acontecimiento importante en intertnet en
general; ``The Drugde Report'' (Sitio web de aquella época) anunció una noticia
antes que la televisión, algo nunca antes visto ya que el internet no era un
medio común para este tipo de contenido. Desde ese momento, se empezó a hacer
un uso más intensivo del internet como medio de divulgación de noticias y como
medio de comunicación en general.

Como resultado de ``The Drugde Report'' (O como el pionero), en 2003, se fundó
MySpace el cual era un sitio de comunicación social. Esto dió paso a la creación
de lo que hoy se conoce como YouTube, por mencionar un ejemplo.

En 2004 sucedieron 3 cosas importantes: el concepto de la ``Web 2.0'' (del cual
se hablará más adelante), lo que hoy conocemos como publicidad en la web (el cual
funcionaba a través de las votaciones de los usuarios) y el lanzamiento de
Facebook, que inicialmente solo era abierto a estudiantes pero que actualmente
es uno de los medios de comunicación más utilizados en todo el planeta.

En 2005 fue lanzado YouTube de manera oficial, plataforma la cual permite
compartir y visualizar videos en línea.

2006 fue el turno de Twitter, también una red social pero con un paradigma muy
diferente al que podemos obtener con Facebook.

En 2007 se lanzó el primer iPhone que fue uno de los grandes logros de la
tecnología, que dió paso al mundo de las apliaciones móviles y que abrió aún
más el paso a las aplicaciones web, siendo posible así, acceder a estas últimas
desde un smartphone.

\subsection{Futuro de las aplicaciones web}
Como se menciona en \cite{um}, el desarrollo de aplicaciones web parece ser muy
prometedor. Esto puede ser debido a que han sido sobrepasadas las barreras que
impedían que esto fuese factible y viable, como pueden ser la velocidad de
carga/descarga o la facilidad de acceso a internet.

Actualmente existen aplicaciones que son capaces de realizar las mismas tareas
que una aplicación de escritorio o una aplicación móvil, como edición de
documentos, marketing, tiendas online, comunicación, etc.

\subsection{Navegadores}
A principios de 1990, en los laborarios del CERN, a mismo tiempo que se
desarrollaba la www (World Wide Web), también se estaba desarrollando HTML
(Hypertext Markup Language) el cual permite dialogar entre los sitios y fue
utilizado inicialmente para poder compartir información entre los involucrados
en este campo. Teniendo ya creado tanto la www como HTML solo faltaba el
navegador\cite{navs}.

El primer navegador creado fue Mosaic, lanzado en Estados Unidos en 1994, el cual
se encontraba en condiciones para poder comunicar los puertos de un
dispositivo y poder así acceder a los documentos de interés.

El siguiente navegador creado fue Netscape. Este navegador tiene una gran
importancia en la historia ya que permitió llevar la www al usuario común, ya
que anteriormente estos recursos solo estaban disponibles para universitarios
o invetigadores.

Seguido de esto surgió Internet Explorer, el navegador creado por Microsoft.
Gracias a esto se generó una ``guerra'' sobre los nevegadores ya que Microsoft
quería dominar este sector y que su producto fuese el más utilizado.

En 1995 se dió a conocer Opera, un navegador el cual se tenían la intención de
que fuese el más rápido y ligero de todos.

La primera versión de Firefox fue lanzada en 2002. Este ganó el corazón de
muchas personas que estaban a favor del software libre y destacó por su
ligereza, versatilidad y por ser un sistema basado en extensiones\cite{navs}.

\subsection{Webs}

\subsubsection{Web 1.0}
Esta web se considera solo de lectura ya que es la más baśica y la mayoría del
contenido es únicamente texto, es decir, el usuario no tiene ninguna interación
con el sitio web, siemplemente puede acceder al contenido de este \cite{webs}.

Caracterítiscas:
\begin{itemize}
	\item Páginas estáticas
	\item Uso de framesets o marcos
	\item Formularios HTML
	\item Las páginas se actualizaba poco y eran fijas
\end{itemize}

\subsubsection{Web 2.0}
El diseño se enfoca en el usuario, facilita compartir información,
interoperabilidad. Es esta web se permite la interacción tanto con el sitio
como con otros usuarios \cite{webs}.

Ejemplos de servicios:
\begin{itemize}
	\item Blogs
	\item Wikis
	\item Redes sociales
	\item Documentos
	\item Videos
	\item Fotos
	\item Presentaciones
	\item Plataformas educativas
\end{itemize}

\subsubsection{Web 3.0}
Describe la interacción entre las personas e internet. Está enfocado sobre todo
al uso de la inteligencia artifical, la web semántica, web geoespacial y la
web 3D \cite{webs}.

\subsubsection{Web 4.0}
En esta web se pretende que se pase a una red aún más inteligente donde las
``cosas'' puedan comunicarse con las personas y tomar decisiones en base a los
datos que estas proporcionan.
Esta web aún está en desarrollo pero todo indica que es el futuro de la web
y la forma (en algunos años) en la que dispositivos como computadores, smartphones
refrigeradores, cafeteras y otros dispositivos electrónicos sean inteligentes
y que así pueden intercambiar información con las personas de una forma natural \cite{webs}.

\section{Conclusiones}
Las web es un tipo de tecnología que es increíblemente conveniente; cualquier
dispositivo con acceso a internet y un navegador web actual es capaz de acceder
a este contenido, comenzando desde smartphones hasta computadoras. Esto
supone una gran ventaja sobre desarrollar software para una plataforma específica.
Cuando se desarrolla un sistema para una plataforma en concreto, la empresa u
organización interesada en brindar dicho servicio tendrá que desarrollar el mismo
software para cada una de ellas, es decir, tendría que desarrollar una
aplicación para android, una para ios, otra para macOS, Windows, Linux, etc.

Las aplicaciones web son el futuro de la tecnología; internet estás en todas
partes y con la web 4.0 en todas las cosas. Esto permite una interacción más
natural entre la tecnología como dispositivos electrónicos, las
personas y el intercambio de información entre ellos. No está en duda que en
algunos años la tecnología esté tan integrada en la vida de las personas que no
se pueda distinguir entre personas y dispositivos compartiendo información, será
tan natural que las personas olvidarán que hay algo más funcionando detrás
que hace que sus vidas sea más productivas, fáciles o eficientes.

Otra razón por la que las aplicaciones web son una opción a considerar, es que
actualmente aquellas limitaciones que hacían imposible tener aplicaciones web
corriendo de manera eficiente en dispositivos móviles han desaparecido con el
paso del tiempo; las personas tiene acceso a internet la mayoría del tiempo, los
dispositivos electrónicos son cada vez más rápidos, la velocidad de carga y
descarga ha aumentado de manera exarada a comparación de años atrás y con la
próxima llegada de la tecnología 5G pareciera ser que no habría límites ni
excusas por las cuales las aplicaciones web no deban ser la primera opción a
considerar a la hora de desarrollar software que brinde algún tipo de servicio
a los usuarios.

\begin{thebibliography}{}
    \bibitem{wiki_web_app}
	``Web application''. [online]
	Available at: https://en.wikipedia.org/wiki/Web\_application
	[Accessed 27 Jan. 2019].
    \bibitem{um}
	R. Barzanallana. ``Servicios en internet Historia del desarrollo de aplicaciones Web''.
	[online] Available at: https://www.um.es/docencia/barzana/DIVULGACION/INFORMATICA/Historia-desarrollo-aplicaciones-web.html
	[Accessed 27 Jan. 2019].
    \bibitem{wiki_perl}
	``Perl''. [online]
	Available at: https://es.wikipedia.org/wiki/Perl
	[Accessed 27 Jan. 2019].
    \bibitem{navs}
	``Breve historia de los navegadores''. [online]
	Available at: http://www.consumer.es/web/es/tecnologia/internet
	/2005/10/26/146455.php
	[Accessed 27 Jan. 2019].
    \bibitem{webs}
	``Evolución de las Aplicaciones Web''. [online]
	Available at: http://appsdelweb.blogspot.com/2013/02/11-evolucion-de-las-aplicaciones-web.html
	[Accessed 27 Jan. 2019].
\end{thebibliography}

\end{document}
