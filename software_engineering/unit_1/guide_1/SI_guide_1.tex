\documentclass{article}

%Configuraciones de idioma del documento
\usepackage[spanish]{babel}
%Configurar entrada {usar caracteres desde el teclado como á}
\usepackage[utf8]{inputenc}
%Fuente que pueda renderizar caracteres en nuestro idioma
\usepackage[T1]{fontenc}
%Fuente compatible
\usepackage{lmodern}

\begin{document}

\title{Guía de Ingeniería de Software Unidad 1}
\author{Antonio Emiko Ochoa Adame}
\maketitle

\section{Introducción}

\textbf{Requisito}.- Propiedad que debe ser exhibida por un software para
resolver un problema. (Se necesita para resolver un problema)

\vspace{1em}
\textbf{Requisito}.- Condición o capacidad que necesita el usuario para resolver
un problema o conseguir un objetivo determinado. (Lo necesita usuario para
conseguir sus objetivos).

\vspace{1em}
\textbf{Ingeniería de requisitos}.- Conjunto de actividades para descubrir,
documentar y mantener un conjunto de requisitos. (Actividades para mantener bien
los requisitos).

\vspace{1em}
\textbf{Ingeniería de requisitos}.- Establecer los servicios que el cliente
requiere de un sistema y las restricciones bajo las cuales opera y es desarrollado.
(Restricciones y cosas necesarias para que se desarrolle el software).

\vspace{1em}
\textbf{Proceso de ingeniería de requisitos}.- Conjunto estructurado de
actividades de cuya ejecución se obtiene, valida y mantiene el documento
de requisitos de un sistema. (Proceso que se lleva para mantener correcto el
documento de requisitos).

\vspace{1em}
\textbf{Gestión de requisitos}.- Actividad para gestionar los cambios en los
requisitos de un sistema. (Gestión de cambios de requisitos).

\section{Revisión de especificación de requerimientos}

\textbf{¿Qué implican los requisitos?}

\begin{itemize}
	\item
		Costo 15\%-20\% del total del proyecto.
	\item
		Error en requisitos es 100 veces peor que en error de código.
	\item
		Un error en requisitos genera errores en las demás etapas.
	\item
		Necesita gestión y soporte automatizado.
\end{itemize}

\subsection{Especificación de requeriemtos (ERS)}

Es la descripción completa del comportamiento del sistema que se va a desarrollar.
Aquí están involucrados los requisitos funcionales (casos de uso) y los
requisitos no funcionales (complementarios; restricciones en el diseño o en la
implementación).

\vspace{1em}
Técnicas de levantamiento de requerimientos:
\begin{itemize}
	\item
		Entrevistas
	\item
		Prototipos
	\item
		Casos de uso
	\item
		Cuestionarios
	\item
		Lluvia de ideas
	\item
		Etc.
\end{itemize}

\subsection{Requerimientos según algunos estándares}

\textbf{IEEE 830}

Indica los elementos que debe tener una especificación de requerimientos.

\vspace{1em}
\textbf{PMBOK}

Indica los procesos para la gestión de proyectos.

\vspace{1em}
\textbf{BABOK}

Indica los procesos para el análisis del negocio.

\subsection{Norma IEEE 830}

Está enfocado en recomendaciones prácticas para la especificación de requerimientos.

Ayuda a describir de forma precisa qué quieren en el software (para el cliente)
y establecer una estructura estándar para la ERS (para el desarrollador).


\begin{itemize}
	\item
		Ayuda a elaborar el SRS (Software Requirements Specification) o
		ERS (Especificación de Requerimientos de Software).
	\item
		Guía de redacción de ERS (no es obligatorio).
	\item
		Hecha en 1998 por la Software Engineering Standards Commitee del
		IEEE Computer Society.
\end{itemize}

\textbf{¿Para qué sirve esta norma?}

\begin{itemize}
	\item
		Para saber claramente lo que el cliente quiere
	\item
		Para establecer bases de un contrato
	\item
		Para reducir análisis, diseño y programación
	\item
		Para tener una referencia para validación de software
	\item
		Permite hacer mejoras o innovaciones
\end{itemize}

\vspace{1em}
\textbf{¿Quién la debe utilizar?}

\begin{itemize}
	\item
		Un cliente
	\item
		Un desarrollador de ``software a la medida''
	\item
		Un desarrollador de ``software de paquete''
\end{itemize}

\subsection{Trazabilidad de requisitos}

Permite relacionar y establecer \textbf{dependencias} entre los requisitos y con
otrso elementos importantes del proyecto (diseño, componentes, testting, realeases
etc.).

\vspace{1em}
\textbf{Beneficios}

\begin{itemize}
	\item
		Gestión óptima del alcance de la solución
	\item
		Gestionar cambios con mínimo impacto
	\item
		Ayuda a reducir riesgos en el proyecto
	\item
		Ayuda a manetener consistencia entre requisitos
	\item
		Permite monitorizar y controlar el ciclo de vida de los
		requerimientos
\end{itemize}

\vspace{1em}
\textbf{Matriz de trazabilidad}

\vspace{1em}

\section{Descripción de procesos actuales (Modelado de negocioModelado de negocioss)}

Se hace mediante diagramas de flujo utilizando la metodología sipoc.

\vspace{1em}
Diagrama sipoc consta de:

\begin{itemize}
	\item
		Supliers (Proveedores)
	\item
		Inputs (Entradas)
	\item
		Process (Procesos)
	\item
		Output (Salidas)
	\item
		Customers (Clientes)
\end{itemize}

\vspace{1em}
\textbf{Proceso}.- Serie de tareas o actividades interrelacionadas para alcanzar
un determinado fin.

\vspace{1em}
\textbf{Análisis de procesos}.- Permite entender y medir las actividades de un
proceso.

\vspace{1em}
\textbf{Modelado de procesos}.- Requiere habilidades y técnicas que permitan
comprender, comunicar, medir y gestionar los principales componentes de los
procesos más importantes del negocio.

\section{Estudio de factibilidad}

Antes de proceder con un desarrollo debe evaluarse su viabilidad y analizar los
riesgos que implica.

\vspace{1em}
Aspectos a considerar:

\begin{itemize}
	\item
		Económico (El beneficio compensa los costo?, vale la pena invertir?)
	\item
		Técnico (Funcionalidad, rendimiento o restricciones realizables?)
	\item
		Legal (Requisitos rompen ley o regla?)
	\item
		Operativa (Se puede implantar de manera efectiva en la empresa?,
		encaja en la filosofía de la empresa?, el personal está motivado
		a usar el software?)
	\item
		Plazos y calendario (El plazo es realista?, fecha elegidas son
		apropiadas?)
\end{itemize}

\vspace{1em}
Hay 6 fases en el estudio de factibilidad.

\section{Análisis costo-beneficio}

Permite saber si una inversión vale la pena o no para el negocio.

\vspace{1em}
Cuando se considera una desición, el costo de dicha desición se le resta al
beneficio de la misma.

\vspace{1em}
Requiere mucha experiencia técnica y de gestión.

\vspace{1em}
Se deben considerar los elementos\textbf{tangibles} (se pueden valorar directamente)
y los \textbf{intangibles} (subjetivos).

\vspace{1em}
\textbf{Tangibles}.- Gastos de equipo, tiempo empleado, etc.

\vspace{1em}
\textbf{Intangibles}.- Mayor competitividad, ventajas económicas, etc.

\section{Roles}

\subsection{Administrador del proyecto}

Administra y controla los recursos asignados a un proyecto para que se cumplan
correctamente las metas establecidas.

\subsection{Analista}

Se encargan de estudiar un problema de complejidad determinada, descomponiendo
el problema en subproblemas de menor complejidad.

\subsection{Diseñador}

Encargado de generar el diseño del sistema.

\vspace{1em}
Sus funciones:

\begin{itemize}
	\item
		Generar el diseño arquitectónico y diseño detallado del sistema,
		basándose en los requisitos
	\item
		Generar prototipos rápidos
	\item
		Generar el documento de diseño
	\item
		Asegurarse de que el producto final se apegue al diseño realizado
\end{itemize}

\subsection{Programador}

Debe convertir la especificación del sistema en código fuente ejecutable
utilizando uno o más lenguajes de programación, así como herramientas de
software de apoyo a la programación.

\subsection{Téster}

Objetivos:

\begin{itemize}
	\item
		Construir y aplicar los planes de prueba unitarios, de módulo,
		de sistema, y aceptación parcial, manteniéndolos actualizados
		durante el proyecto
	\item
		Velar por la adhesión al estándar adoptado para el desarrollo
	\item
		Velar por la calidad del producto final (cumplimiento de los requisitos)
	\item
		Coordinar las inspecciones
\end{itemize}

\subsection{Asegurador de calidad}

Se encarga de que la calidad del software no se vea comprometida debido a la
presión por cumplir con las fechas o por tratar de reducir los costos, entre
otros.


\vspace{1em}
\textbf{Disclaimer}: La finalidad de este documento es servir como apoyo de estudio.
El autor de la versión original de este documento no se hace responsable del
uso indebido del mismo.

\end{document}
